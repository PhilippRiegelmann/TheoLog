\documentclass[aspectratio=1610,onlymath]{beamer}
% \documentclass[aspectratio=1610,onlymath,handout]{beamer}

% Macros used by all lectures, but not necessarily by excercises

% Common notation

\usepackage{amsmath,amssymb,amsfonts}
\usepackage{xspace}

\newcommand{\lectureurl}{https://iccl.inf.tu-dresden.de/web/TheoLog2024}

\DeclareMathAlphabet{\mathsc}{OT1}{cmr}{m}{sc} % Let's have \mathsc since the slide style has no working \textsc

% Dual of "phantom": make a text that is visible but intangible
\newcommand{\ghost}[1]{\raisebox{0pt}[0pt][0pt]{\makebox[0pt][l]{#1}}}

\newcommand{\tuple}[1]{\langle{#1}\rangle}
\newcommand{\defeq}{\mathrel{:=}}

%%% Annotation %%%

\usepackage{color}
\newcommand{\todo}[1]{{\tiny\color{red}\textbf{TODO: #1}}}


\newcommand{\quantor}{\mathord{\reflectbox{$\text{\sf{Q}}$}}} % the generic quantor
\newcommand{\unieq}{\stackrel{.}{=}} % equality sign for unification problems


%%% Old macros below; move when needed

\newcommand{\blank}{\text{\textvisiblespace}} % empty tape cell for TM

% table syntax
\newcommand{\dom}{\textbf{dom}}
\newcommand{\adom}{\textbf{adom}}
\newcommand{\dbconst}[1]{\texttt{"#1"}}
\newcommand{\pred}[1]{\textsf{#1}}
\newcommand{\foquery}[2]{#2[#1]}
\newcommand{\ground}[1]{\textsf{ground}(#1)}
% \newcommand{\foquery}[2]{\{#1\mid #2\}} %% Notation as used in Alice Book
% \newcommand{\foquery}[2]{\tuple{#1\mid #2}}

% logic syntax
\newcommand{\Inter}{\mathcal{I}} %used to denote an interpretation
\newcommand{\Jnter}{\mathcal{J}} %used to denote another interpretation
\newcommand{\Knter}{\mathcal{K}} %used to denote yet another interpretation
\newcommand{\Zuweisung}{\mathcal{Z}} %used to denote a variable assignment

% query languages
\newcommand{\qlang}[1]{{\sf #1}} % Font for query languages
\newcommand{\qmaps}[1]{\textbf{QM}({\sf #1})} % Set of query mappings for a query language

%%% Complexities %%%

\hyphenation{Exp-Time} % prevent "Ex-PTime" (see, e.g. Tobies'01, Glimm'07 ;-)
\hyphenation{NExp-Time} % better that than something else

% \newcommand{\complclass}[1]{{\sc #1}\xspace} % font for complexity classes
\newcommand{\complclass}[1]{\ensuremath{\mathsc{#1}}\xspace} % font for complexity classes

\newcommand{\ACzero}{\complclass{AC$_0$}}
\newcommand{\LogSpace}{\complclass{L}}
\newcommand{\NLogSpace}{\complclass{NL}}
\newcommand{\PTime}{\complclass{P}}
\newcommand{\NP}{\complclass{NP}}
\newcommand{\coNP}{\complclass{coNP}}
\newcommand{\PH}{\complclass{PH}}
\newcommand{\PSpace}{\complclass{PSpace}}
\newcommand{\NPSpace}{\complclass{NPSpace}}
\newcommand{\ExpTime}{\complclass{ExpTime}}
\newcommand{\NExpTime}{\complclass{NExpTime}}
\newcommand{\ExpSpace}{\complclass{ExpSpace}}
\newcommand{\TwoExpTime}{\complclass{2ExpTime}}
\newcommand{\NTwoExpTime}{\complclass{N2ExpTime}}
\newcommand{\ThreeExpTime}{\complclass{3ExpTime}}
\newcommand{\kExpTime}[1]{\complclass{#1ExpTime}}
\newcommand{\kExpSpace}[1]{\complclass{#1ExpSpace}}


%%% Style commands

\newcommand{\quoted}[1]{\texttt{"}{#1}\texttt{"}}
\newcommand{\squote}{\texttt{"}} % straight quote
\newcommand{\Sterm}[1]{\ensuremath{\mathtt{\textcolor{purple}{#1}}}}    % letters in alphabets
\newcommand{\Snterm}[1]{\textsf{\textcolor{darkblue}{#1}}} % nonterminal symbols
\newcommand{\Sntermsub}[2]{\ensuremath{\Snterm{#1}_{\Snterm{#2}}}} % nonterminal symbols
\newcommand{\Slang}[1]{\textbf{\textcolor{black}{#1}}}    % languages
\newcommand{\Slangsub}[2]{\ensuremath{\Slang{#1}_{\Slang{#2}}}}    % languages
% Code
\newcommand{\Scode}[1]{\textbf{#1}}    % reserved words in program listings, e.g., "if"
\newcommand{\Scodelit}[1]{\textcolor{purple}{#1}}    % literals in program listings, e.g., strings
\newcommand{\Scomment}[1]{\textcolor{gray}{#1}}    % comment in program listings
% LOOP and WHILE programs
\newcommand{\Svar}[1]{\texttt{#1}} % variable names
\newcommand{\Svsub}[2]{\ensuremath{\Svar{#1}_{\Svar{#2}}}} % variable names
\newcommand{\Sxsub}[1]{\Svsub{x}{#1}} % variable names
\newcommand{\Sseq}{\texttt{;}}
\newcommand{\SStartLoop}[1]{\Scode{LOOP}~#1~\Scode{DO}~}
\newcommand{\SEndLoop}{\Scode{END}}
\newcommand{\SStartIf}[1]{\Scode{IF}~#1~\Scode{THEN}~}
\newcommand{\SElse}{\Scode{ELSE}}
\newcommand{\SEndIf}{\Scode{END}}
\newcommand{\Svassign}{\texttt{ := }} % variable assignments
\newcommand{\Svneq}{\texttt{!=}\ensuremath{\,}} % inequality operator
\newcommand{\Splus}{\texttt{ + }} % addition operator
\newcommand{\Sminus}{\texttt{ - }} % subtraction operator
\newcommand{\SStartWhile}[1]{\Scode{WHILE}~#1\Svneq\texttt{\Scodelit{0}}~\Scode{DO}~}
\newcommand{\SEndWhile}{\Scode{END}}

\newcommand{\bbfunc}{\boldsymbol{\Sigma}}

\newcommand{\epstrastar}{\mathrel{\mathord{\stackrel{\epsilon}{\to}}{}^*}} % transitive reflexive closure of epsilon transitions in an epslion-NFA

\newcommand{\narrowcentering}[1]{\mbox{}\hfill#1\hfill\mbox{}}

\newcommand{\Smach}[1]{\ensuremath{\mathcal{#1}}}    % machines

\newcommand{\mytrue}{\Scodelit{1}}
\newcommand{\myfalse}{\Scodelit{0}}
% \newcommand{\emptyClause}{\bot}

\newcommand{\Scomplclass}[1]{{\textsc{#1}}\xspace} % font for complexity classes, used on slides where the "too many alphabets" LaTeX error appears when using the correct sc font :-(
% \newcommand{\complclass}[1]{\ensuremath{\mathsc{#1}}} % font for complexity classes


\DeclareMathOperator{\enc}{\mathsf{enc}}

\let\complclass\Scomplclass % prevent "too many math alphabets" error on some slides

%%% General setup and dependencies:

% \usetheme[ddcfooter,nosectionnum]{tud}
\usetheme[nosectionnum,pagenum,noheader]{tud}
% \usetheme[nosectionnum,pagenum]{tud}
\setbeamertemplate{enumerate items}[default]

% Increase body font size to a sane level:
\let\origframetitle\frametitle
% \renewcommand{\frametitle}[1]{\origframetitle{#1}\normalsize}
\renewcommand{\frametitle}[1]{\origframetitle{#1}\fontsize{10pt}{13.2}\selectfont}
\setbeamerfont{itemize/enumerate subbody}{size=\small} % tud defaults to scriptsize!
\setbeamerfont{itemize/enumerate subsubbody}{size=\small}
% \setbeamerfont{normal text}{size=\small}
% \setbeamerfont{itemize body}{size=\small}

\renewcommand{\emph}[1]{\textbf{#1}}

\def\arraystretch{1.3}% Make tables even less cramped vertically

\usepackage[ngerman]{babel}
\usepackage[utf8]{inputenc}
\usepackage[T1]{fontenc}

%\usepackage{graphicx}
\usepackage[export]{adjustbox} % loads graphicx
\usepackage{import}
\usepackage{stmaryrd}
\usepackage[normalem]{ulem} % sout command
% \usepackage{times}
\usepackage{txfonts}
\usepackage{array}

% \usepackage[perpage]{footmisc} % reset footnote counter on each page -- fails with beamer (footnotes gone)
\usepackage{perpage}  % reset footnote counter on each page
\MakePerPage{footnote}

\usepackage{tikz}
\usetikzlibrary{arrows,positioning,decorations.pathreplacing}
% Inspired by http://www.texample.net/tikz/examples/hand-drawn-lines/
\usetikzlibrary{decorations.pathmorphing}
\pgfdeclaredecoration{penciline}{initial}{
    \state{initial}[width=+\pgfdecoratedinputsegmentremainingdistance,
    auto corner on length=1mm,]{
        \pgfpathcurveto%
        {% From
            \pgfqpoint{\pgfdecoratedinputsegmentremainingdistance}
                      {\pgfdecorationsegmentamplitude}
        }
        {%  Control 1
        \pgfmathrand
        \pgfpointadd{\pgfqpoint{\pgfdecoratedinputsegmentremainingdistance}{0pt}}
                    {\pgfqpoint{-\pgfdecorationsegmentaspect
                     \pgfdecoratedinputsegmentremainingdistance}%
                               {\pgfmathresult\pgfdecorationsegmentamplitude}
                    }
        }
        {%TO
        \pgfpointadd{\pgfpointdecoratedinputsegmentlast}{\pgfpoint{1pt}{1pt}}
        }
    }
    \state{final}{}
}
\tikzset{handdrawn/.style={decorate,decoration=penciline}}
\tikzset{every shadow/.style={fill=none,shadow xshift=0pt,shadow yshift=0pt}}
% \tikzset{module/.append style={top color=\col,bottom color=\col}}

% Use to make Tikz attributes with Beamer overlays
% http://tex.stackexchange.com/a/6155
\tikzset{onslide/.code args={<#1>#2}{%
  \only<#1| handout:0>{\pgfkeysalso{#2}}
}}
\tikzset{onslideprint/.code args={<#1>#2}{%
  \only<#1>{\pgfkeysalso{#2}}
}}

%%% Title -- always set this first

\newcommand{\defineTitle}[3]{
	\newcommand{\lectureindex}{#1}
	\title{Theoretische Informatik und Logik}
	\subtitle{\href{\lectureurl}{#1. Vorlesung: #2}}
	\author{\href{https://iccl.inf.tu-dresden.de/web/Markus_Kr\%C3\%B6tzsch}{Markus Kr\"{o}tzsch}\\[1ex]Professur Wissensbasierte Systeme}
	\date{#3}
	\datecity{TU Dresden}
% 	\institute{CC-By 3.0, sofern keine anderslautenden Bildrechte angegeben sind}
}

%%% Table of contents:

\RequirePackage{ifthen}

\newcommand{\highlight}[2]{%
	\ifthenelse{\equal{#1}{\lectureindex}}{\alert{#2}}{#2}%
}

\def\myspace{-0.7ex}
\newcommand{\printtoc}{
\begin{tabular}{r@{$\quad$}l}
\highlight{1}{1.} & \highlight{1}{Willkommen/Einleitung formale Sprachen}\\[\myspace]
\highlight{2}{2.} & \highlight{2}{Grammatiken und die Chomsky-Hierarchie}\\[\myspace]
\highlight{3}{3.} & \highlight{3}{Endliche Automaten}\\[\myspace]
\highlight{4}{4.} & \highlight{4}{Complexity of FO query answering}\\[\myspace]
\highlight{5}{5.} & \highlight{5}{Conjunctive queries}\\[\myspace]
\highlight{6}{6.} & \highlight{6}{Tree-like conjunctive queries}\\[\myspace]
\highlight{7}{7.} & \highlight{7}{Query optimisation}\\[\myspace]
\highlight{8}{8.} & \highlight{8}{Conjunctive Query Optimisation / First-Order~Expressiveness}\\[\myspace]
\highlight{9}{9.} & \highlight{9}{First-Order~Expressiveness / Introduction to Datalog}\\[\myspace]
\highlight{10}{10.} & \highlight{10}{Expressive Power and Complexity of Datalog}\\[\myspace]
\highlight{11}{11.} & \highlight{11}{Optimisation and Evaluation of Datalog}\\[\myspace]
\highlight{12}{12.} & \highlight{12}{Evaluation of Datalog (2)}\\[\myspace]
\highlight{13}{13.} & \highlight{13}{Graph Databases and Path Queries}\\[\myspace]
\highlight{14}{14.} & \highlight{14}{Outlook: database theory in practice}
\end{tabular}
}

\newcommand{\overviewslide}{%
\begin{frame}\frametitle{Overview}
\printtoc
\medskip

Siehe \href{\lectureurl}{course homepage [$\Rightarrow$ link]} for more information and materials
\end{frame}
}

%%% Colours:
\usepackage{xcolor,colortbl}
\definecolor{redhighlights}{HTML}{FFAA66}
\definecolor{lightblue}{HTML}{55AAFF}
\definecolor{lightred}{HTML}{FF5522}
\definecolor{lightpurple}{HTML}{DD77BB}
\definecolor{lightgreen}{HTML}{55FF55}
\definecolor{lightgray}{HTML}{CCCCCC}
\definecolor{darkred}{HTML}{CC4411}
\definecolor{darkblue}{HTML}{176FC0}%{1133AA}
\definecolor{nightblue}{HTML}{2010A0}%{1133AA}
\definecolor{alert}{HTML}{176FC0}
\definecolor{darkgreen}{HTML}{36AB14}
\definecolor{strongyellow}{HTML}{FFE219}
\definecolor{devilscss}{HTML}{666666}

\newcommand{\redalert}[1]{\textcolor{darkred}{#1}}

%%% Slide layout commands:

\newcommand{\sectionSlide}[1]{
\frame{\begin{center}
\LARGE
#1
\end{center}}
}
\newcommand{\sectionSlideNoHandout}[1]{
\frame<handout:0>{\begin{center}
\LARGE
#1
\end{center}}
}

\newcommand{\mydualbox}[3]{%
 \begin{minipage}[t]{#1}
 \begin{beamerboxesrounded}[upper=block title,lower=block body,shadow=true]%
    {\centering\usebeamerfont*{block title}#2}%
    \raggedright%
    \usebeamerfont{block body}
%     \small
    #3%
  \end{beamerboxesrounded}
  \end{minipage}
}
%
\newcommand{\myheaderbox}[2]{%
 \begin{minipage}[t]{#1}
 \begin{beamerboxesrounded}[upper=block title,lower=block title,shadow=true]%
    {\centering\usebeamerfont*{block title}\rule{0pt}{2.6ex} #2}%
  \end{beamerboxesrounded}
  \end{minipage}
}

\newcommand{\mycontentbox}[2]{%
 \begin{minipage}[t]{#1}%
 \begin{beamerboxesrounded}[upper=block body,lower=block body,shadow=true]%
    {\centering\usebeamerfont*{block body}\rule{0pt}{2.6ex}#2}%
  \end{beamerboxesrounded}
  \end{minipage}
}

\newcommand{\mylcontentbox}[2]{%
 \begin{minipage}[t]{#1}%
 \begin{beamerboxesrounded}[upper=block body,lower=block body,shadow=true]%
    {\flushleft\usebeamerfont*{block body}\rule{0pt}{2.6ex}#2}%
  \end{beamerboxesrounded}
  \end{minipage}
}

% label=180:{\rotatebox{90}{{\footnotesize\textcolor{darkgreen}{Beispiel}}}}
% \hspace{-8mm}\ghost{\raisebox{-7mm}{\rotatebox{90}{{\footnotesize\textcolor{darkgreen}{Beispiel}}}}}\hspace{8mm}
\newcommand{\examplebox}[1]{%
	\begin{tikzpicture}[decoration=penciline, decorate]
		\pgfmathsetseed{1235}
		\node (n1) [decorate,draw=darkgreen, fill=darkgreen!10,thick,align=left,text width=\linewidth, inner ysep=2mm, inner xsep=2mm] at (0,0) {#1};
% 		\node (n2) [align=left,text width=\linewidth,inner sep=0mm] at (n1.92) {{\footnotesize\raisebox{3mm}{\textcolor{darkgreen}{Beispiel}}}};
% 		\node (n2) [decorate,draw=darkgreen, fill=darkgreen!10,thick, align=left,text width=\linewidth,inner sep=2mm] at (n1.90) {{\footnotesize\raisebox{0mm}{\textcolor{darkgreen}{Beispiel}}}};
	\end{tikzpicture}%
}%

\newcommand{\codebox}[1]{%
	\begin{tikzpicture}[decoration=penciline, decorate]
		\pgfmathsetseed{1236}
		\node (n1) [decorate,draw=strongyellow, fill=strongyellow!10,thick,align=left,text width=\linewidth, inner ysep=2mm, inner xsep=2mm] at (0,0) {#1};
	\end{tikzpicture}%
}%

\newcommand{\defbox}[1]{%
	\begin{tikzpicture}[decoration=penciline, decorate]
		\pgfmathsetseed{1237}
		\node (n1) [decorate,draw=darkred, fill=darkred!10,thick,align=left,text width=\linewidth, inner ysep=2mm, inner xsep=2mm] at (0,0) {#1};
	\end{tikzpicture}%
}%

\newcommand{\theobox}[1]{%
	\begin{tikzpicture}[decoration=penciline, decorate]
		\pgfmathsetseed{1240}
		\node (n1) [decorate,draw=darkblue, fill=darkblue!10,thick,align=left,text width=\linewidth, inner ysep=2mm, inner xsep=2mm] at (0,0) {#1};
	\end{tikzpicture}%
}%

\newcommand{\anybox}[2]{%
	\begin{tikzpicture}[decoration=penciline, decorate]
		\pgfmathsetseed{1240}
		\node (n1) [decorate,draw=#1, fill=#1!10,thick,align=left,text width=\linewidth, inner ysep=2mm, inner xsep=2mm] at (0,0) {#2};
	\end{tikzpicture}%
}%


\newsavebox{\mybox}%
\newcommand{\doodlebox}[2]{%
\sbox{\mybox}{#2}%
	\begin{tikzpicture}[decoration=penciline, decorate]
		\pgfmathsetseed{1238}
		\node (n1) [decorate,draw=#1, fill=#1!10,thick,align=left,inner sep=1mm] at (0,0) {\usebox{\mybox}};
	\end{tikzpicture}%
}%


\defineTitle{13}{Prädikatenlogik: Syntax und Semantik}{10. Juni 2021}

\begin{document}

\maketitle

\sectionSlide{Halbzeit:\\Zusammenfassung und Ausblick}

\begin{frame}\frametitle{Überblick Berechenbarkeit}

\begin{center}
\alert{Vorlesung 1}\\
Hilbert in Paris $\bullet$ Die Turingmaschine $\bullet$ Church-Turing-These\\[1ex]
\alert{Vorlesung 2}\\
Berechenbarkeit $\bullet$ Entscheidbarkeit $\bullet$ Busy Beaver \\[1ex]
\alert{Vorlesung 3}\\LOOP $\bullet$ WHILE $\bullet$ Turing-Mächtigkeit von WHILE \\[1ex]
\alert{Vorlesung 4}\\Die Kraft des LOOP $\bullet$ Halteproblem $\bullet$ Reduktionen \\[1ex]
\alert{Vorlesung 5}\\Satz von Rice $\bullet$ Semi-Entscheidbarkeit\\ Postsches Korrespondenzproblem \\[1ex]
\alert{Vorlesung 6}\\Probleme formaler Sprachen $\bullet$ Noch unentscheidbarere Probleme\\ Euklid als Informatiker
\end{center}

\end{frame}

\begin{frame}\frametitle{Überblick Komplexität}

\begin{center}
\alert{Vorlesung 7}\\
Königsberger Brücken $\bullet$ Beschränkte TMs $\bullet$ Robuste Komplexität\ghost{en}\\[1ex]
\alert{Vorlesung 8}\\
Zeit vs. Raum $\bullet$ \complclass{P} und \complclass{L}  $\bullet$ Polynomielle Reduktionen \\[1ex]
\alert{Vorlesung 9}\\
Coles Beweis ohne Worte $\bullet$ \complclass{NP}\\$\Slang{SAT}\leq_p \Slang{Clique}\leq_p \Slang{Unabhänige Menge}$ \\[1ex]
\alert{Vorlesung 10}\\
$\Slang{SAT}\leq_p \Slang{Teilmengen Summe}\leq_p \Slang{Rucksack}$ \\Pseudopolynomialität $\bullet$ Fake News zu \complclass{NP} \\[1ex]
\alert{Vorlesung 11}\\
\complclass{NL} und Erreichbarkeit $\bullet$ Schach ist schwer $\bullet$ \PSpace und QBF \\[1ex]
\alert{Vorlesung 12}\\
$\Slang{TrueQBF}\leq_p \Slang{TrueQBF}_{\Slang{alt}}\leq_p \Slang{Geography}$ $\bullet$ Spiele 
%$\bullet$ Evaluation
\end{center}

\end{frame}

\begin{frame}\frametitle{Ausblick: Prädikatenlogik}

\begin{center}
\alert{Vorlesung 13}\\
Logik erster Stufe: Syntax und Semantik\\[1ex]
\alert{Vorlesung 14}\\
Modelltheorie $\bullet$ Logisches Schließen\\[1ex]
\alert{Vorlesung 15}\\
Gleichheit $\bullet$ Unentscheidbarkeit der Schließens \\[1ex]
\alert{Vorlesung 16}\\
Ableitungskalküle $\bullet$ syntaktische Umformungen \\[1ex]
\alert{Vorlesung 17}\\
Normalformen $\bullet$ Funktionssymbole $\bullet$ Skolemisierung\\[1ex]
\alert{Vorlesung 18}\\
Unifikation $\bullet$ Substitutionen \\[1ex]
\alert{Vorlesung 19}\\
Resolution $\bullet$ Herbrand $\bullet$ Löwenheim-Skolem

\end{center}

\end{frame}

\begin{frame}\frametitle{Ausblick: Logik und mehr}

\begin{center}
\alert{Vorlesung 20}\\
Korrektheit der Resolution $\bullet$ endliche Modelle\\[1ex]
\alert{Vorlesung 21}\\
Datalog $\bullet$ Logikprogrammierung \\[1ex]
\alert{Vorlesung 22}\\
Logik höherer Ordnung $\bullet$ entscheidbare Fragmente\\[1ex]
\alert{Vorlesung 23}\\
Logik auf natürlichen Zahlen $\bullet$ Gödel\\[1ex]
\alert{Vorlesung 24}\\
Gödel \& Turing $\bullet$ Zusammenfassung $\bullet$ Ausblick
\end{center}

\end{frame}

\sectionSlide{Prädikatenlogik}

\begin{frame}\frametitle{Logik}

\begin{center}
Aristoteles (384--324 v.Chr.):\\[1cm]

"`Alle Menschen sind sterblich."'\\
"`Sokrates ist ein Mensch."'\\
\rule{7cm}{1pt}

"`Also ist Sokrates sterblich."'

\end{center}

\end{frame}

\begin{frame}\frametitle{Gut, aber in der Informatik?}

Wozu lehrt man Informatiker:innen Logik?
\begin{itemize}
\item \alert{Verifikation von Software:} Logik als Spezifikationssprache gewünschter Eigenschaften
\item \alert{Künstliche Intelligenz:} Logik als Lehre vom folgerichtigen Denken
\item \alert{Optimierung:} Logisches Schließen als Suche nach Lösungen komplexer Probleme
\item \alert{Datenverwaltung:} Logik als eindeutig definiertes Format für Daten und Anfragen
\item \alert{Wissensrepräsentation:} Logik zur Darstellung und Anwendung von menschlichem Wissen in Form von Regeln, Taxonomien und Ontologien
\end{itemize}

\end{frame}

\begin{frame}\frametitle{Rückblick: Syntax der Aussagenlogik}

In der Aussagenlogik gibt es eine abzählbar unendliche Menge $\Slang{P}$ von
\redalert{atomaren Aussagen} (auch bekannt als: \redalert{aussagenlogische Variablen}, \redalert{Propositionen} oder schlicht \redalert{Atome})
\medskip

\defbox{Die Menge der \redalert{aussagenlogischen Formeln} ist induktiv\footnote{Das bedeutet: Die Definition ist selbstbezüglich und soll die kleinste\\ Menge an Formeln beschreiben, die alle Bedingungen erfüllen.} definiert:
\begin{itemize}
\item Jedes Atom $p\in\Slang{P}$ ist eine aussagenlogische Formel
\item Wenn $F$ und $G$ aussagenlogische Formeln sind, so auch:
	\begin{itemize}
	\item $\neg F$: \redalert{Negation}, "`nicht $F$"'
	\item $(F\wedge G)$: \redalert{Konjunktion}, "`$F$ und $G$"'
	\item $(F\vee G)$: \redalert{Disjunktion}, "`$F$ oder $G$"'
	\item $(F\to G)$: \redalert{Implikation}, "`$F$ impliziert $G$"'
	\item $(F\leftrightarrow G)$: \redalert{Äquivalenz}, "`$F$ ist äquivalent zu $G$"'
	\end{itemize}
\end{itemize}
}

\end{frame}

\begin{frame}\frametitle{Rückblick: Semantik der Aussagenlogik}

Eine aussagenlogische Formel kann wahr oder falsch sein, je nachdem,
wie man die atomaren Aussagen interpretiert. Dazu verwenden wir eine
\redalert{Wertzuweisung} $w:\Slang{P}\to\{\mytrue,\myfalse\}$.
\smallskip

\defbox{Eine Wertzuweisung $w$ \redalert{erfüllt} eine Formel $F$, in Symbolen \redalert{$w\models F$}, wenn
eine der folgenden rekursiven Bedingungen gilt:\smallskip

{\footnotesize
\narrowcentering{\begin{tabular}{rll}
\rowcolor{darkred!70!gray}
\textcolor{white}{Form von $F$} & \textcolor{white}{$w\models F$ wenn:} & \textcolor{white}{$w\not\models F$ wenn:}\\
$F\in \Slang{P}$: & $w(F)=\mytrue$ & $w(F)=\myfalse$\\
\rowcolor{lightred!20}
$F=\neg G$ & $w\not\models G$ & $w\models G$\\
$F=(G_1\wedge G_2)$ & $w\models G_1$ und $w\models G_2$ & $w\not\models G_1$ oder $w\not\models G_2$\\
\rowcolor{lightred!20}
$F=(G_1\vee G_2)$ & $w\models G_1$ oder $w\models G_2$ & $w\not\models G_1$ und $w\not\models G_2$\\
$F=(G_1\to G_2)$ & $w\not\models G_1$ oder $w\models G_2$ & $w\models G_1$ und $w\not\models G_2$\\
\rowcolor{lightred!20}
$F=(G_1\leftrightarrow G_2)$ & $w\models G_1$ und $w\models G_2$ & $w\models G_1$ und $w\not\models G_2$\\%[-1ex]
\rowcolor{lightred!20}
	& \multicolumn{1}{c}{\raisebox{1ex}{oder}} & \multicolumn{1}{c}{\raisebox{1ex}{oder}} \\[-2ex]
\rowcolor{lightred!20}
	& $w\not\models G_1$ und $w\not\models G_2$ & $w\not\models G_1$ und $w\models G_2$\\[-1ex]
\\[-1.5ex]
\end{tabular}}}

Dabei bedeutet "`A oder B"' immer "`A oder B oder beides"'.
}

\end{frame}

% \begin{frame}\frametitle{Wo ist er Sinn?}
% 
% \begin{itemize}
% \item Atomare Aussagen haben keine logische Bedeutung
% \item Wertzuweisungen sind beliebig
% \end{itemize}
% Was kann uns Logik dann überhaupt sagen?
% \pause\bigskip
% 
% Es geht um die Beziehung von \alert{Annahmen} und \alert{Schlussfolgerungen}, d.h. um \alert{logisches Schließen}:
% \begin{itemize}
% \item Allgemein betrachten wir alle möglichen Wertzuweisungen
% \item Wenn wir eine Formel als wahr annehmen, dann beschränken wir damit die möglichen Wertzuweisungen {\tiny da wir nur noch Wertezuweisungen zulassen, welche die Annahem wahr machen}
% \item Je weniger Wertzuweisungen man betrachtet, desto mehr Formeln sind 
% \end{itemize}
% 
% 
% \end{frame}



\begin{frame}\frametitle{Von Aussagen- zu Prädikatenlogik}

Die Prädikatenlogik gibt atomaren Aussagen eine innere Struktur:
\begin{itemize}
\item aus "`$\,p_{\textsf{Soktrates-ist-ein-Mensch}}\!$"' wird
\item "`$\textsf{istMensch}(\textsf{sokrates})$"'
\end{itemize}\bigskip\pause

Dadurch können mehrere Dinge die selbe Eigenschaft haben:
\[ \text{istMensch}(\text{sokrates})\quad \text{istMensch}(\text{emilia})\quad \text{istMensch}(\text{anton})\quad\cdots\]

\pause Für allgemeine Aussagen über die Dinge, die eine Eigenschaft haben, gibt es zwei
\alert{Quantoren:}
\begin{itemize}
\item \redalert{Existenzquantor $\exists$}: Mindestens ein Ding hat die Eigenschaft
\item \redalert{Allquantor $\forall$}: Alle Dinge haben die Eigenschaft
\end{itemize}

\end{frame}

\begin{frame}\frametitle{Beispiele}

\alert{"`Jeder Mensch ist sterblich."'}
\[ \forall x.\big(\textsf{istMensch}(x)\to\textsf{istSterblich}(x)\big)\]

\pause\alert{"`Es ist nicht alles Gold was glänzt."'}\pause
\begin{center}\begin{minipage}{6.5cm}
\begin{enumerate}[(a)]
\item $\forall x.\big(\textsf{glänzt}(x)\wedge\neg\textsf{istGold}(x)\big)$ \onslide<4->{\textcolor{darkred}{~~~falsch}}
\item $\forall x.\big(\textsf{glänzt}(x)\to\neg\textsf{istGold}(x)\big)$ \onslide<4->{\textcolor{darkred}{~~~falsch}}
\item $\neg\forall x.\big(\textsf{glänzt}(x)\to\textsf{istGold}(x)\big)$ \onslide<4->{\textcolor{darkgreen}{~~~richtig}}
\item $\exists x.\big(\textsf{glänzt}(x)\wedge\neg\textsf{istGold}(x)\big)$ \onslide<4->{\textcolor{darkgreen}{~~~richtig}}
\end{enumerate}\end{minipage}\end{center}
% \[ \neg\forall x.\big(\textsf{glänzt}(x)\to\textsf{istGold}(x)\big)\]
% bzw. äquivalent
% \[ \exists x.\big(\textsf{glänzt}(x)\wedge\neg\textsf{istGold}(x)\big)\]

\pause\pause\bigskip
\alert{"`Null ist eine natürliche Zahl und jede natürliche Zahl hat einen Nachfolger, der
ebenfalls eine natürliche Zahl ist."'}
\[ \textsf{NatNum}(\textsf{null})\wedge \forall x.\Big(\textsf{NatNum}(x)\to\exists y.\big(\textsf{succ}(x,y)\wedge \textsf{NatNum}(y)\big)\Big)\]


\end{frame}

\begin{frame}\frametitle{Intuitive Semantik: Logelei}

In einem entlegenen Inselreich gibt es zwei Arten von Menschen:
\begin{itemize}
\item die einen (Typ W) sagen stets die Wahrheit
\item die anderen (Typ L) lügen immer
\end{itemize}
\bigskip\pause

Wir besuchen einige der Inseln und fragen die Einheimischen
nach dem Typ der Bewohner dort.\pause

\begin{itemize}
\item Auf \redalert{Insel A} antwortet jeder der Bewohner:\\ \alert{"`Wir sind hier alle vom gleichen Typ."'}\\
\begin{itemize}
\item Stimmt das? Und wenn ja, von welchem Typ sind sie?
\end{itemize}\pause
\item Auf \redalert{Insel B} antwortet jeder der Bewohner:\\ \alert{"`Hier gibt es einige vom Typ W und einige vom Typ L."'}
\begin{itemize}
\item Was sagt uns das über die Typen der Bewohner von B?
\end{itemize}
\end{itemize}




\end{frame}

\begin{frame}\frametitle{Prädikatenlogik: Syntax (1)}

In der Aussagenlogik gab es eine unendliche Menge von Atomen.\pause 
\medskip

In der Prädikatenlogik betrachten wir stattdessen mehrere Mengen:%

\anybox{strongyellow}{\vspace{-1ex}
\begin{itemize}
\item Eine Menge $\Slang{V}$ von \redalert{Variablen} $x$, $y$, $z$, \ldots
\item Eine Menge $\Slang{C}$ von \redalert{Konstanten} $a$, $b$, $c$, \ldots
\item Eine Menge $\Slang{P}$ von \redalert{Prädikatensymbolen} $p$, $q$, $r$, \ldots \\
jedes Prädikat hat eine \redalert{Stelligkeit} $\geq 0$ (auch \redalert{Arität} genannt)
\end{itemize}
Variablen und Konstanten werden auch \redalert{Terme} genannt.}

Diese Mengen sind:
\begin{enumerate}[(1)]
\item \alert{abzählbar unendlich} (d.h. wir haben immer ausreichend Symbole jeden Typs
zur Verfügung)
\item \alert{disjunkt} (d.h. es ist stets klar, welcher Art ein Symbol ist)
\end{enumerate}\pause 

\defbox{Ein prädikatenlogisches \redalert{Atom} ist ein Ausdruck $p(t_1,\ldots,t_n)$
für ein $n$-stelliges Prädikatensymbol $p\in\Slang{P}$ und Terme $t_1,\ldots,t_n\in\Slang{V}\cup\Slang{C}$.}

\end{frame}

\begin{frame}\frametitle{Prädikatenlogik: Syntax (2)}

Formeln werden jetzt wie in der Aussagenlogik aus Atomen aufgebaut:

\defbox{Die Menge der \redalert{prädikatenlogischen Formeln} ist induktiv definiert:
\begin{itemize}
\item Jedes Atom $p(t_1,\ldots,t_n)$ ist eine prädikatenlogische Formel
\item Wenn $x\in\Slang{V}$ eine Variable und $F$ und $G$ prädikatenlogische Formeln sind, dann sind auch die folgenden prädikatenlogische Formeln:
	\begin{itemize}
	\item $\neg F$: \redalert{Negation}, "`nicht $F$"'
	\item $(F\wedge G)$: \redalert{Konjunktion}, "`$F$ und $G$"'
	\item $(F\vee G)$: \redalert{Disjunktion}, "`$F$ oder $G$"'
	\item $(F\to G)$: \redalert{Implikation}, "`$F$ impliziert $G$"'
	\item $(F\leftrightarrow G)$: \redalert{Äquivalenz}, "`$F$ ist äquivalent zu $G$"'
	\item $\exists x. F$: \redalert{Existenzquantor}, "`für ein $x$ gilt $F$"'
	\item $\forall x. F$: \redalert{Allquantor}, "`für alle $x$ gilt $F$"'
	\end{itemize}
\end{itemize}
}

\end{frame}
\begin{frame}\frametitle{Funktionssymbole}

\emph{Anmerkung:} Oft werden in der Prädikatenlogik Funktionssymbole
betrachtet, die in Termen (zusätzlich zu Variablen und Konstanten)
vorkommen dürfen.
\bigskip

Wir verzichten hier vorerst darauf, da
\begin{itemize}
\item dies die Definition komplexer (und eventuell verwirrender) macht,
\item man damit keine zusätzliche Ausdruckstärke erhält und
\item wir diese Erweiterung für spezielle Anwendungen leicht auch später noch einführen können.
\end{itemize}

\end{frame}
\begin{frame}\frametitle{Konventionen und Vereinfachungen}

Wir erlauben folgende \alert{Vereinfachungen der Syntax:}\pause
\begin{enumerate}[(1)]
\item Bei nullstelligen Prädikatensymbolen lassen wir die (leeren) Klammern weg.\\
Beispiel: wir schreiben $(p\to q)$ anstelle von $(p()\to q())$\pause
%
\item Die äußersten Klammern einer Formel dürfen weggelassen werden.\\
Beispiel: wir schreiben $p\to q$ statt $(p\to q)$\pause
%
\item Klammern innerhalb geschachtelter Konjunktionen und Disjunktionen dürfen
weggelassen werden.\\
Beispiel: wir schreiben $p\wedge q\wedge r$ statt $p\wedge (q\wedge r)$\pause
%
\item Hintereinander vorkommende Quantoren gleicher Art werden zusammengefasst, wobei man die Variablen als Liste angibt.\\
Beispiel: wir schreiben $\forall x,x'.\exists y,y'.(p(x,x')\to p(y,y'))$ statt $\forall x.\forall x'.\exists y.\exists y'.(p(x,x')\to p(y,y'))$\pause
\end{enumerate}

\redalert{Ansonsten schreiben wir Formel immer gemäß der formalen Syntax, einschließlich aller Klammern.}

\end{frame}


\begin{frame}\frametitle{Teilformeln}

\defbox{Die \redalert{Teilformeln} einer Formel sind alle Teilausdrücke
der Formel, welche selbst Formeln sind.}

Die Menge der Teilformeln ist eindeutig bestimmt, wenn man der formalen 
Definition der Formel folgt.

\examplebox{\emph{Beispiel:} Die Teilformeln von $\forall x.(p(x)\vee (q(x)\vee r(x)))$ sind
$(p(x)\vee (q(x)\vee r(x)))$, $(q(x)\vee r(x))$, $p(x)$, $q(x)$, $r(x)$ sowie die Formel selbst, aber nicht $p(x)\vee (q(x)$ oder $\forall x.(p(x)$.}

\pause\emph{Anmerkung:} Lässt man manche Klammern zur Vereinfachung der Syntax weg, dann 
sind die Teilformeln nicht mehr eindeutig bestimmt.

\examplebox{\emph{Beispiel:} $\forall x.(p(x)\vee q(x)\vee r(x))$  könnte sowohl
$\forall x.(p(x)\vee (q(x)\vee r(x)))$ als auch $\forall x.((p(x)\vee q(x))\vee r(x))$
bezeichnen, was zu unterschiedlichen Teilformeln führt.}

\end{frame}

\begin{frame}\frametitle{Freie und gebundene Variablenvorkommen}

Jedes Vorkommen einer Variable in einer Formel ist entweder \alert{frei} oder \alert{gebunden}:

\defbox{Die \redalert{freien Vorkommen} von Variablen in einer Formel sind rekursiv wie folgt definiert:
\begin{itemize}
\item In einem Atom sind alle Vorkommen von Variablen frei.
\item Die freien Vorkommen in $\neg F$ sind die selben wie in $F$
\item Die freien Vorkommen in $(F\wedge G)$, $(F\vee G)$, $(F\to G)$ und
$(F\leftrightarrow G)$ sind alle freien Vorkommen aus $F$ und $G$
\item Die freien Vorkommen in $\forall x.F$ und $\exists x.F$ sind die 
selben wie in $F$, ohne die Vorkommen von $x$ in $F$
\end{itemize}
Damit sind Vorkommen von Variablen $x$ genau dann \redalert{gebunden}
wenn sie im Bereich eines Quantors auftauchen.
}\pause

\examplebox{\emph{Beispiel:} Eine Variable kann gleichzeitig frei und gebunden in einer Formel 
auftauchen, wie z.B. $x$ in $p(x)\wedge \exists x.q(x)$.}

\end{frame}

\begin{frame}\frametitle{Offene und geschlossene Formeln}

\defbox{Formeln ohne freie Vorkommen von Variablen heißen\\\redalert{geschlossene Formeln} oder \redalert{Sätze}.}

\bigskip\emph{Anmerkung:} Eigentlich würde man sich oft gern auf Sätze beschränken (da sie 
geschlossene Aussagen darstellen).

\smallskip
Leider muss man sich dennoch mit
offenen Formeln beschäftigen, weil diese auftauchen, sobald man die Teilformeln
einer geschlossenen Formel anschaut (dies tun wir in  vielen Definitionen und
Beweisen).\bigskip

\defbox{Formeln mit freien Vorkommen von Variablen heißen \redalert{offene Formeln}.}


\end{frame}

\sectionSlide{Semantik der Prädikatenlogik}

\begin{frame}\frametitle{Atome interpretieren}

\emph{Grundgedanke:} Der Wahrheitswert von Formeln sollte sich 
wie in der Aussagenlogik aus dem Wahrheitswert von Atomen ergeben.
\bigskip

\examplebox{\emph{Beispiel:} Ein mögliches Atom ist $p(a,x)$ mit $a\in\Slang{C}$ und $x\in\Slang{V}$. Wie soll diesem Ausdruck ein Wahrheitswert zugeordnet werden?}

\alert{Dazu müssen drei Fragen beantwortet werden:}
\begin{itemize}
\item Was bedeutet $p$?\\
	\visible<2->{Prädikatssymbole stehen für Relationen gleicher Stelligkeit}
\item Was bedeutet $a$?\\
	\visible<3->{Konstanten stehen für Elemente, die in Relationen stehen \ghost{können}}
\item Was bedeutet $x$?\\
	\visible<4->{Variablen stehen auch für Elemente, aber ihre verschiedenen Vorkommen können für verschiedene Elemente stehen}
\end{itemize}

\visible<5->{\alert{
Ein Atom $p(a,x)$ ist also wahr, wenn zwischen dem Element, für das $a$ steht, und dem Element, für das $x$ steht, die Relation besteht, für die $p$ steht.}}

\end{frame}

\begin{frame}\frametitle{Interpretationen}

Die Wertzuweisung der Aussagenlogik wird also durch \alert{Interpretationen}
und \alert{Zuweisungen} für Variablen ersetzt.
\smallskip

\defbox{Eine \redalert{Interpretation} $\Inter$ ist ein Paar $\tuple{\Delta^\Inter,\cdot^\Inter}$ bestehend aus einer nichtleeren Grundmenge von Elementen $\Delta^\Inter$ (der \redalert{Domäne}) und einer \redalert{Interpretationsfunktion} $\cdot^\Inter$, welche:
\begin{itemize}
\item jede Konstante $a\in\Slang{C}$ auf ein Element $a^\Inter\in\Delta^\Inter$ und
\item jedes $n$-stellige Prädikatensymbol $p\in\Slang{P}$ auf eine Relation $p^\Inter\subseteq(\Delta^\Inter)^n$
\end{itemize}
abbildet.
}

\defbox{Eine \redalert{Zuweisung} $\Zuweisung$ für eine Interpretation $\Inter$ ist eine Funktion $\Zuweisung:\Slang{V}\to\Delta^\Inter$, die Variablen auf Elemente der Domäne abbildet. Für $x\in\Slang{V}$ und $\delta\in\Delta^\Inter$ schreiben wir \redalert{$\Zuweisung [x\mapsto \delta]$} für die Zuweisung, die $x$ auf $\delta$ und alle anderen Variablen $y\neq x$ auf $\Zuweisung(y)$ abbildet.}

\end{frame}

\begin{frame}[t]\frametitle{Beispiel (1)}

\alert{"`Null ist eine natürliche Zahl und jede natürliche Zahl hat einen Nachfolger, der
ebenfalls eine natürliche Zahl ist."'}
\[ \textsf{NatNum}(\textsf{null})\wedge \forall x.\Big(\textsf{NatNum}(x)\to\exists y.\big(\textsf{succ}(x,y)\wedge \textsf{NatNum}(y)\big)\Big)\]

Wir betrachten eine Interpretation $\Inter$ mit
\begin{itemize}
\item $\Delta^\Inter=\mathbb{R}$ die Menge der reellen Zahlen
\item $\textsf{null}^\Inter=0$
\item $\textsf{NatNum}^\Inter=\mathbb{N}\subseteq\mathbb{R}$ die Menge der natürlichen Zahlen
\item $\textsf{succ}^\Inter=\{\tuple{d,e}\mid d,e\in\mathbb{R},d<e\}$
\end{itemize}

Unter dieser Interpretation ist das Atom $\textsf{NatNum}(\textsf{null})$ wahr, da
$\textsf{null}^\Inter\in\textsf{NatNum}^\Inter$ gilt.
\smallskip

Wir schreiben: $\textsf{NatNum}(\textsf{null})^\Inter=\mytrue$.

\end{frame}

\begin{frame}[t]\frametitle{Beispiel (2)}

\alert{"`Null ist eine natürliche Zahl und jede natürliche Zahl hat einen Nachfolger, der
ebenfalls eine natürliche Zahl ist."'}
\[ \textsf{NatNum}(\textsf{null})\wedge \forall x.\Big(\textsf{NatNum}(x)\to\exists y.\big(\textsf{succ}(x,y)\wedge \textsf{NatNum}(y)\big)\Big)\]

Wir betrachten für die vorherige Interpretation $\Inter$ 
eine Zuweisung $\Zuweisung$ mit
\begin{itemize}
\item $\Zuweisung(x)=42$
\item $\Zuweisung(y)=5$
\end{itemize}

Unter dieser Interpretation und Zuweisung ist das Atom $\textsf{succ}(x,y)$ falsch, da
$\tuple{\Zuweisung(x),\Zuweisung(y)}\notin\textsf{succ}^\Inter$ gilt.
\smallskip

Wir schreiben: $\textsf{succ}(x,y)^{\Inter,\Zuweisung}=\myfalse$.
\bigskip

\anybox{strongyellow}{\emph{Anmerkung:} Allgemein geben wir bei Interpretationen und Zuweisungen
nur die Belegungen an, die für die Auswertung einer gegebenen Formel relevant sind.
Zum Beispiel müssen wir $\Zuweisung$ nicht für alle $z\in\Slang{V}$ definieren.
}

\end{frame}

\begin{frame}\frametitle{Atome interpretieren}

Wir bestimmen dementsprechend die Wahrheit von Atomen unter einer Interpretation und
Zuweisung:

\defbox{Sei $\Inter$ eine Interpretation und $\Zuweisung$ eine Zuweisung für $\Inter$.
\begin{itemize}
\item Für eine Konstante $c$ definieren wir $c^{\Inter,\Zuweisung}=c^\Inter$
\item Für eine Variable $x$ definieren wir $x^{\Inter,\Zuweisung}=\Zuweisung(x)$
\end{itemize}
Für ein Atom $p(t_1,\ldots,t_n)$ setzen wir sodann:
\begin{itemize}
\item $p(t_1,\ldots,t_n)^{\Inter,\Zuweisung}=\mytrue$ wenn $\tuple{t_1^{\Inter,\Zuweisung},\ldots,t_n^{\Inter,\Zuweisung}}\in p^\Inter$ und
\item $p(t_1,\ldots,t_n)^{\Inter,\Zuweisung}=\myfalse$ wenn $\tuple{t_1^{\Inter,\Zuweisung},\ldots,t_n^{\Inter,\Zuweisung}}\notin p^\Inter$.
\end{itemize}
}\bigskip\pause

\anybox{strongyellow}{
\emph{Achtung!} Wir verwenden Interpretationen und Zuweisungen auf zwei Ebenen, 
die man nicht verwechseln sollte:
\begin{enumerate}[(1)]
\item um Terme $t$ auf Elemente $t^{\Inter,\Zuweisung}\in\Delta^\Inter$ abzubilden
\item um Atome $A$ auf Wahrheitswerte $A^{\Inter,\Zuweisung}\in\{\myfalse,\mytrue\}$ abzubilden
\end{enumerate}}

\end{frame}

\begin{frame}\frametitle{Formeln interpretieren}

% Die Wahrheit ganzer Formeln ergibt sich nun aus der Wahrheit ihrer Atome:

\defbox{Eine Interpretation $\Inter$ und eine Zuweisung $\Zuweisung$ für
$\Inter$ \redalert{erfüllen} eine Formel $F$, in Symbolen \redalert{$\Inter,\Zuweisung\models F$}, wenn
eine der folgenden rekursiven Bedingungen gilt:%\smallskip

\narrowcentering{\footnotesize\begin{tabular}{rll}
\rowcolor{darkred!70!gray}
\textcolor{white}{Formel $F$} & \textcolor{white}{$\Inter,\Zuweisung\models F$ wenn:} & \textcolor{white}{$\Inter,\Zuweisung\not\models F$ wenn:}\\
$F$ Atom & $F^{\Inter,\Zuweisung}=\mytrue$ & $F^{\Inter,\Zuweisung}=\myfalse$\\
\rowcolor{lightred!20}
$\neg G$ & $\Inter,\Zuweisung\not\models G$ & $\Inter,\Zuweisung\models G$\\
$(G_1\wedge G_2)$ & $\Inter,\Zuweisung\models G_1$ und $\Inter,\Zuweisung\models G_2$ & $\Inter,\Zuweisung\not\models G_1$ oder $\Inter,\Zuweisung\not\models G_2$\\
\rowcolor{lightred!20}
$(G_1\vee G_2)$ & $\Inter,\Zuweisung\models G_1$ oder $\Inter,\Zuweisung\models G_2$ & $\Inter,\Zuweisung\not\models G_1$ und $\Inter,\Zuweisung\not\models G_2$\\
$(G_1\,{\to}\, G_2)$ & $\Inter,\Zuweisung\not\models G_1$ oder $\Inter,\Zuweisung\models G_2$ & $\Inter,\Zuweisung\models G_1$ und $\Inter,\Zuweisung\not\models G_2$\\
\rowcolor{lightred!20}
$(G_1\,{\leftrightarrow}\,G_2)$ & $\Inter,\Zuweisung\models G_1$ und $\Inter,\Zuweisung\models G_2$ & $\Inter,\Zuweisung\models G_1$ und $\Inter,\Zuweisung\not\models G_2$\\%[-1ex]
\rowcolor{lightred!20}
	& \multicolumn{1}{c}{\raisebox{1ex}{oder}} & \multicolumn{1}{c}{\raisebox{1ex}{oder}} \\[-2ex]
\rowcolor{lightred!20}
	& $\Inter,\Zuweisung\not\models G_1$ und $\Inter,\Zuweisung\not\models G_2$ & $\Inter,\Zuweisung\not\models G_1$ und $\Inter,\Zuweisung\models G_2$\\
%
$\forall x.G$ & $\Inter,\Zuweisung[x\mapsto\delta]\models G$ & $\Inter,\Zuweisung[x\mapsto\delta]\not\models G$\\[-0.5ex]
 & für alle $\delta\in\Delta^\Inter$ & für mindestens ein $\delta\in\Delta^\Inter$
\\
%
\rowcolor{lightred!20}
$\exists x.G$ & $\Inter,\Zuweisung[x\mapsto\delta]\models G$ & $\Inter,\Zuweisung[x\mapsto\delta]\not\models G$\\[-0.5ex]
\rowcolor{lightred!20}
 & für mindestens ein $\delta\in\Delta^\Inter$ & für alle $\delta\in\Delta^\Inter$
	\\[-1ex]
\\[-3.25ex]
\end{tabular}}

~
}

\end{frame}

\begin{frame}\frametitle{Beispiel}

\alert{"`Null ist eine natürliche Zahl und jede natürliche Zahl hat einen Nachfolger, der
ebenfalls eine natürliche Zahl ist."'}
\[ F=\textsf{NatNum}(\textsf{null})\wedge \forall x.\Big(\textsf{NatNum}(x)\to\exists y.\big(\textsf{succ}(x,y)\wedge \textsf{NatNum}(y)\big)\Big)\]

Wir betrachten eine Interpretation $\Inter$ mit
\begin{itemize}
\item $\Delta^\Inter=\mathbb{R}$ die Menge der reellen Zahlen
\item $\textsf{null}^\Inter=0$
\item $\textsf{NatNum}^\Inter=\mathbb{N}\subseteq\mathbb{R}$ die Menge der natürlichen Zahlen
\item $\textsf{succ}^\Inter=\{\tuple{d,e}\mid d,e\in\mathbb{R},d<e\}$
\end{itemize}\bigskip

Dann gilt $\Inter\models F$ (unter jeder beliebigen Zuweisung).

\anybox{strongyellow}{\emph{Notation:} Bei der Interpretation von Sätzen (Formeln ohne freie Variablen) spielen Zuweisungen keine Rolle. Wir schreiben sie in diesem Fall nicht.}

\end{frame}

\begin{frame}\frametitle{Intuitive Semantik: Logelei}

Wir kehren zurück auf das Inselreich mit Menschen von
Typ W und Typ L.
\bigskip

Smullyan\footnote{R. Smullyan: A Beginner's Guide to Mathematical Logic, Dover 2014} fragte die Bewohner nach ihren Rauchgewohnheiten.\pause

\begin{itemize}
\item Auf \redalert{Insel A} antwortete jeder der Bewohner:\\ \alert{"`Jeder, der hier von Typ W ist, raucht."'}\\
%
\item Auf \redalert{Insel B} antwortete jeder der Bewohner:\\ \alert{"`Einige von uns hier sind von Typ L und rauchen."'}
%
\item Auf \redalert{Insel C} hatten alle den gleichen Typ und jeder sagte:\\ \alert{"`Falls ich rauche, dann raucht jeder hier."'}
%
\item Auf \redalert{Insel D} hatten alle den gleichen Typ und jeder sagte:\\ \alert{"`Einige hier rauchen, aber ich nicht."'}
\end{itemize}

Was können wir jeweils über die Bewohner und ihre Gewohnheiten schließen?

\end{frame}



\begin{frame}\frametitle{Zusammenfassung und Ausblick}

Prädikatenlogik verallgemeinert Aussagenlogik mit $n$-stelligen Prädikaten, Termen und Quantoren
\bigskip

Zur Auswertung von Formeln benötigen wir eine Interpretation und eine Variablenzuweisung
\bigskip

% Spiele liefern interessante Beispiele komplexer Probleme
% \bigskip

\anybox{yellow}{
Was erwartet uns als nächstes?
\begin{itemize}
\item Logisches Schließen
\item Entscheidbare logische Probleme
\item Unentscheidbare logische Probleme
\end{itemize}
}

\end{frame}



\end{document}
