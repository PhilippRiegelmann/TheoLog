\documentclass[aspectratio=1610,onlymath]{beamer}
% \documentclass[aspectratio=1610,onlymath,handout]{beamer}

% Macros used by all lectures, but not necessarily by excercises

% Common notation

\usepackage{amsmath,amssymb,amsfonts}
\usepackage{xspace}

\newcommand{\lectureurl}{https://iccl.inf.tu-dresden.de/web/TheoLog2024}

\DeclareMathAlphabet{\mathsc}{OT1}{cmr}{m}{sc} % Let's have \mathsc since the slide style has no working \textsc

% Dual of "phantom": make a text that is visible but intangible
\newcommand{\ghost}[1]{\raisebox{0pt}[0pt][0pt]{\makebox[0pt][l]{#1}}}

\newcommand{\tuple}[1]{\langle{#1}\rangle}
\newcommand{\defeq}{\mathrel{:=}}

%%% Annotation %%%

\usepackage{color}
\newcommand{\todo}[1]{{\tiny\color{red}\textbf{TODO: #1}}}


\newcommand{\quantor}{\mathord{\reflectbox{$\text{\sf{Q}}$}}} % the generic quantor
\newcommand{\unieq}{\stackrel{.}{=}} % equality sign for unification problems


%%% Old macros below; move when needed

\newcommand{\blank}{\text{\textvisiblespace}} % empty tape cell for TM

% table syntax
\newcommand{\dom}{\textbf{dom}}
\newcommand{\adom}{\textbf{adom}}
\newcommand{\dbconst}[1]{\texttt{"#1"}}
\newcommand{\pred}[1]{\textsf{#1}}
\newcommand{\foquery}[2]{#2[#1]}
\newcommand{\ground}[1]{\textsf{ground}(#1)}
% \newcommand{\foquery}[2]{\{#1\mid #2\}} %% Notation as used in Alice Book
% \newcommand{\foquery}[2]{\tuple{#1\mid #2}}

% logic syntax
\newcommand{\Inter}{\mathcal{I}} %used to denote an interpretation
\newcommand{\Jnter}{\mathcal{J}} %used to denote another interpretation
\newcommand{\Knter}{\mathcal{K}} %used to denote yet another interpretation
\newcommand{\Zuweisung}{\mathcal{Z}} %used to denote a variable assignment

% query languages
\newcommand{\qlang}[1]{{\sf #1}} % Font for query languages
\newcommand{\qmaps}[1]{\textbf{QM}({\sf #1})} % Set of query mappings for a query language

%%% Complexities %%%

\hyphenation{Exp-Time} % prevent "Ex-PTime" (see, e.g. Tobies'01, Glimm'07 ;-)
\hyphenation{NExp-Time} % better that than something else

% \newcommand{\complclass}[1]{{\sc #1}\xspace} % font for complexity classes
\newcommand{\complclass}[1]{\ensuremath{\mathsc{#1}}\xspace} % font for complexity classes

\newcommand{\ACzero}{\complclass{AC$_0$}}
\newcommand{\LogSpace}{\complclass{L}}
\newcommand{\NLogSpace}{\complclass{NL}}
\newcommand{\PTime}{\complclass{P}}
\newcommand{\NP}{\complclass{NP}}
\newcommand{\coNP}{\complclass{coNP}}
\newcommand{\PH}{\complclass{PH}}
\newcommand{\PSpace}{\complclass{PSpace}}
\newcommand{\NPSpace}{\complclass{NPSpace}}
\newcommand{\ExpTime}{\complclass{ExpTime}}
\newcommand{\NExpTime}{\complclass{NExpTime}}
\newcommand{\ExpSpace}{\complclass{ExpSpace}}
\newcommand{\TwoExpTime}{\complclass{2ExpTime}}
\newcommand{\NTwoExpTime}{\complclass{N2ExpTime}}
\newcommand{\ThreeExpTime}{\complclass{3ExpTime}}
\newcommand{\kExpTime}[1]{\complclass{#1ExpTime}}
\newcommand{\kExpSpace}[1]{\complclass{#1ExpSpace}}


%%% Style commands

\newcommand{\quoted}[1]{\texttt{"}{#1}\texttt{"}}
\newcommand{\squote}{\texttt{"}} % straight quote
\newcommand{\Sterm}[1]{\ensuremath{\mathtt{\textcolor{purple}{#1}}}}    % letters in alphabets
\newcommand{\Snterm}[1]{\textsf{\textcolor{darkblue}{#1}}} % nonterminal symbols
\newcommand{\Sntermsub}[2]{\ensuremath{\Snterm{#1}_{\Snterm{#2}}}} % nonterminal symbols
\newcommand{\Slang}[1]{\textbf{\textcolor{black}{#1}}}    % languages
\newcommand{\Slangsub}[2]{\ensuremath{\Slang{#1}_{\Slang{#2}}}}    % languages
% Code
\newcommand{\Scode}[1]{\textbf{#1}}    % reserved words in program listings, e.g., "if"
\newcommand{\Scodelit}[1]{\textcolor{purple}{#1}}    % literals in program listings, e.g., strings
\newcommand{\Scomment}[1]{\textcolor{gray}{#1}}    % comment in program listings
% LOOP and WHILE programs
\newcommand{\Svar}[1]{\texttt{#1}} % variable names
\newcommand{\Svsub}[2]{\ensuremath{\Svar{#1}_{\Svar{#2}}}} % variable names
\newcommand{\Sxsub}[1]{\Svsub{x}{#1}} % variable names
\newcommand{\Sseq}{\texttt{;}}
\newcommand{\SStartLoop}[1]{\Scode{LOOP}~#1~\Scode{DO}~}
\newcommand{\SEndLoop}{\Scode{END}}
\newcommand{\SStartIf}[1]{\Scode{IF}~#1~\Scode{THEN}~}
\newcommand{\SElse}{\Scode{ELSE}}
\newcommand{\SEndIf}{\Scode{END}}
\newcommand{\Svassign}{\texttt{ := }} % variable assignments
\newcommand{\Svneq}{\texttt{!=}\ensuremath{\,}} % inequality operator
\newcommand{\Splus}{\texttt{ + }} % addition operator
\newcommand{\Sminus}{\texttt{ - }} % subtraction operator
\newcommand{\SStartWhile}[1]{\Scode{WHILE}~#1\Svneq\texttt{\Scodelit{0}}~\Scode{DO}~}
\newcommand{\SEndWhile}{\Scode{END}}

\newcommand{\bbfunc}{\boldsymbol{\Sigma}}

\newcommand{\epstrastar}{\mathrel{\mathord{\stackrel{\epsilon}{\to}}{}^*}} % transitive reflexive closure of epsilon transitions in an epslion-NFA

\newcommand{\narrowcentering}[1]{\mbox{}\hfill#1\hfill\mbox{}}

\newcommand{\Smach}[1]{\ensuremath{\mathcal{#1}}}    % machines

\newcommand{\mytrue}{\Scodelit{1}}
\newcommand{\myfalse}{\Scodelit{0}}
% \newcommand{\emptyClause}{\bot}

\newcommand{\Scomplclass}[1]{{\textsc{#1}}\xspace} % font for complexity classes, used on slides where the "too many alphabets" LaTeX error appears when using the correct sc font :-(
% \newcommand{\complclass}[1]{\ensuremath{\mathsc{#1}}} % font for complexity classes


\DeclareMathOperator{\enc}{\mathsf{enc}}

\let\complclass\Scomplclass % prevent "too many math alphabets" error on some slides

%%% General setup and dependencies:

% \usetheme[ddcfooter,nosectionnum]{tud}
\usetheme[nosectionnum,pagenum,noheader]{tud}
% \usetheme[nosectionnum,pagenum]{tud}
\setbeamertemplate{enumerate items}[default]

% Increase body font size to a sane level:
\let\origframetitle\frametitle
% \renewcommand{\frametitle}[1]{\origframetitle{#1}\normalsize}
\renewcommand{\frametitle}[1]{\origframetitle{#1}\fontsize{10pt}{13.2}\selectfont}
\setbeamerfont{itemize/enumerate subbody}{size=\small} % tud defaults to scriptsize!
\setbeamerfont{itemize/enumerate subsubbody}{size=\small}
% \setbeamerfont{normal text}{size=\small}
% \setbeamerfont{itemize body}{size=\small}

\renewcommand{\emph}[1]{\textbf{#1}}

\def\arraystretch{1.3}% Make tables even less cramped vertically

\usepackage[ngerman]{babel}
\usepackage[utf8]{inputenc}
\usepackage[T1]{fontenc}

%\usepackage{graphicx}
\usepackage[export]{adjustbox} % loads graphicx
\usepackage{import}
\usepackage{stmaryrd}
\usepackage[normalem]{ulem} % sout command
% \usepackage{times}
\usepackage{txfonts}
\usepackage{array}

% \usepackage[perpage]{footmisc} % reset footnote counter on each page -- fails with beamer (footnotes gone)
\usepackage{perpage}  % reset footnote counter on each page
\MakePerPage{footnote}

\usepackage{tikz}
\usetikzlibrary{arrows,positioning,decorations.pathreplacing}
% Inspired by http://www.texample.net/tikz/examples/hand-drawn-lines/
\usetikzlibrary{decorations.pathmorphing}
\pgfdeclaredecoration{penciline}{initial}{
    \state{initial}[width=+\pgfdecoratedinputsegmentremainingdistance,
    auto corner on length=1mm,]{
        \pgfpathcurveto%
        {% From
            \pgfqpoint{\pgfdecoratedinputsegmentremainingdistance}
                      {\pgfdecorationsegmentamplitude}
        }
        {%  Control 1
        \pgfmathrand
        \pgfpointadd{\pgfqpoint{\pgfdecoratedinputsegmentremainingdistance}{0pt}}
                    {\pgfqpoint{-\pgfdecorationsegmentaspect
                     \pgfdecoratedinputsegmentremainingdistance}%
                               {\pgfmathresult\pgfdecorationsegmentamplitude}
                    }
        }
        {%TO
        \pgfpointadd{\pgfpointdecoratedinputsegmentlast}{\pgfpoint{1pt}{1pt}}
        }
    }
    \state{final}{}
}
\tikzset{handdrawn/.style={decorate,decoration=penciline}}
\tikzset{every shadow/.style={fill=none,shadow xshift=0pt,shadow yshift=0pt}}
% \tikzset{module/.append style={top color=\col,bottom color=\col}}

% Use to make Tikz attributes with Beamer overlays
% http://tex.stackexchange.com/a/6155
\tikzset{onslide/.code args={<#1>#2}{%
  \only<#1| handout:0>{\pgfkeysalso{#2}}
}}
\tikzset{onslideprint/.code args={<#1>#2}{%
  \only<#1>{\pgfkeysalso{#2}}
}}

%%% Title -- always set this first

\newcommand{\defineTitle}[3]{
	\newcommand{\lectureindex}{#1}
	\title{Theoretische Informatik und Logik}
	\subtitle{\href{\lectureurl}{#1. Vorlesung: #2}}
	\author{\href{https://iccl.inf.tu-dresden.de/web/Markus_Kr\%C3\%B6tzsch}{Markus Kr\"{o}tzsch}\\[1ex]Professur Wissensbasierte Systeme}
	\date{#3}
	\datecity{TU Dresden}
% 	\institute{CC-By 3.0, sofern keine anderslautenden Bildrechte angegeben sind}
}

%%% Table of contents:

\RequirePackage{ifthen}

\newcommand{\highlight}[2]{%
	\ifthenelse{\equal{#1}{\lectureindex}}{\alert{#2}}{#2}%
}

\def\myspace{-0.7ex}
\newcommand{\printtoc}{
\begin{tabular}{r@{$\quad$}l}
\highlight{1}{1.} & \highlight{1}{Willkommen/Einleitung formale Sprachen}\\[\myspace]
\highlight{2}{2.} & \highlight{2}{Grammatiken und die Chomsky-Hierarchie}\\[\myspace]
\highlight{3}{3.} & \highlight{3}{Endliche Automaten}\\[\myspace]
\highlight{4}{4.} & \highlight{4}{Complexity of FO query answering}\\[\myspace]
\highlight{5}{5.} & \highlight{5}{Conjunctive queries}\\[\myspace]
\highlight{6}{6.} & \highlight{6}{Tree-like conjunctive queries}\\[\myspace]
\highlight{7}{7.} & \highlight{7}{Query optimisation}\\[\myspace]
\highlight{8}{8.} & \highlight{8}{Conjunctive Query Optimisation / First-Order~Expressiveness}\\[\myspace]
\highlight{9}{9.} & \highlight{9}{First-Order~Expressiveness / Introduction to Datalog}\\[\myspace]
\highlight{10}{10.} & \highlight{10}{Expressive Power and Complexity of Datalog}\\[\myspace]
\highlight{11}{11.} & \highlight{11}{Optimisation and Evaluation of Datalog}\\[\myspace]
\highlight{12}{12.} & \highlight{12}{Evaluation of Datalog (2)}\\[\myspace]
\highlight{13}{13.} & \highlight{13}{Graph Databases and Path Queries}\\[\myspace]
\highlight{14}{14.} & \highlight{14}{Outlook: database theory in practice}
\end{tabular}
}

\newcommand{\overviewslide}{%
\begin{frame}\frametitle{Overview}
\printtoc
\medskip

Siehe \href{\lectureurl}{course homepage [$\Rightarrow$ link]} for more information and materials
\end{frame}
}

%%% Colours:
\usepackage{xcolor,colortbl}
\definecolor{redhighlights}{HTML}{FFAA66}
\definecolor{lightblue}{HTML}{55AAFF}
\definecolor{lightred}{HTML}{FF5522}
\definecolor{lightpurple}{HTML}{DD77BB}
\definecolor{lightgreen}{HTML}{55FF55}
\definecolor{lightgray}{HTML}{CCCCCC}
\definecolor{darkred}{HTML}{CC4411}
\definecolor{darkblue}{HTML}{176FC0}%{1133AA}
\definecolor{nightblue}{HTML}{2010A0}%{1133AA}
\definecolor{alert}{HTML}{176FC0}
\definecolor{darkgreen}{HTML}{36AB14}
\definecolor{strongyellow}{HTML}{FFE219}
\definecolor{devilscss}{HTML}{666666}

\newcommand{\redalert}[1]{\textcolor{darkred}{#1}}

%%% Slide layout commands:

\newcommand{\sectionSlide}[1]{
\frame{\begin{center}
\LARGE
#1
\end{center}}
}
\newcommand{\sectionSlideNoHandout}[1]{
\frame<handout:0>{\begin{center}
\LARGE
#1
\end{center}}
}

\newcommand{\mydualbox}[3]{%
 \begin{minipage}[t]{#1}
 \begin{beamerboxesrounded}[upper=block title,lower=block body,shadow=true]%
    {\centering\usebeamerfont*{block title}#2}%
    \raggedright%
    \usebeamerfont{block body}
%     \small
    #3%
  \end{beamerboxesrounded}
  \end{minipage}
}
%
\newcommand{\myheaderbox}[2]{%
 \begin{minipage}[t]{#1}
 \begin{beamerboxesrounded}[upper=block title,lower=block title,shadow=true]%
    {\centering\usebeamerfont*{block title}\rule{0pt}{2.6ex} #2}%
  \end{beamerboxesrounded}
  \end{minipage}
}

\newcommand{\mycontentbox}[2]{%
 \begin{minipage}[t]{#1}%
 \begin{beamerboxesrounded}[upper=block body,lower=block body,shadow=true]%
    {\centering\usebeamerfont*{block body}\rule{0pt}{2.6ex}#2}%
  \end{beamerboxesrounded}
  \end{minipage}
}

\newcommand{\mylcontentbox}[2]{%
 \begin{minipage}[t]{#1}%
 \begin{beamerboxesrounded}[upper=block body,lower=block body,shadow=true]%
    {\flushleft\usebeamerfont*{block body}\rule{0pt}{2.6ex}#2}%
  \end{beamerboxesrounded}
  \end{minipage}
}

% label=180:{\rotatebox{90}{{\footnotesize\textcolor{darkgreen}{Beispiel}}}}
% \hspace{-8mm}\ghost{\raisebox{-7mm}{\rotatebox{90}{{\footnotesize\textcolor{darkgreen}{Beispiel}}}}}\hspace{8mm}
\newcommand{\examplebox}[1]{%
	\begin{tikzpicture}[decoration=penciline, decorate]
		\pgfmathsetseed{1235}
		\node (n1) [decorate,draw=darkgreen, fill=darkgreen!10,thick,align=left,text width=\linewidth, inner ysep=2mm, inner xsep=2mm] at (0,0) {#1};
% 		\node (n2) [align=left,text width=\linewidth,inner sep=0mm] at (n1.92) {{\footnotesize\raisebox{3mm}{\textcolor{darkgreen}{Beispiel}}}};
% 		\node (n2) [decorate,draw=darkgreen, fill=darkgreen!10,thick, align=left,text width=\linewidth,inner sep=2mm] at (n1.90) {{\footnotesize\raisebox{0mm}{\textcolor{darkgreen}{Beispiel}}}};
	\end{tikzpicture}%
}%

\newcommand{\codebox}[1]{%
	\begin{tikzpicture}[decoration=penciline, decorate]
		\pgfmathsetseed{1236}
		\node (n1) [decorate,draw=strongyellow, fill=strongyellow!10,thick,align=left,text width=\linewidth, inner ysep=2mm, inner xsep=2mm] at (0,0) {#1};
	\end{tikzpicture}%
}%

\newcommand{\defbox}[1]{%
	\begin{tikzpicture}[decoration=penciline, decorate]
		\pgfmathsetseed{1237}
		\node (n1) [decorate,draw=darkred, fill=darkred!10,thick,align=left,text width=\linewidth, inner ysep=2mm, inner xsep=2mm] at (0,0) {#1};
	\end{tikzpicture}%
}%

\newcommand{\theobox}[1]{%
	\begin{tikzpicture}[decoration=penciline, decorate]
		\pgfmathsetseed{1240}
		\node (n1) [decorate,draw=darkblue, fill=darkblue!10,thick,align=left,text width=\linewidth, inner ysep=2mm, inner xsep=2mm] at (0,0) {#1};
	\end{tikzpicture}%
}%

\newcommand{\anybox}[2]{%
	\begin{tikzpicture}[decoration=penciline, decorate]
		\pgfmathsetseed{1240}
		\node (n1) [decorate,draw=#1, fill=#1!10,thick,align=left,text width=\linewidth, inner ysep=2mm, inner xsep=2mm] at (0,0) {#2};
	\end{tikzpicture}%
}%


\newsavebox{\mybox}%
\newcommand{\doodlebox}[2]{%
\sbox{\mybox}{#2}%
	\begin{tikzpicture}[decoration=penciline, decorate]
		\pgfmathsetseed{1238}
		\node (n1) [decorate,draw=#1, fill=#1!10,thick,align=left,inner sep=1mm] at (0,0) {\usebox{\mybox}};
	\end{tikzpicture}%
}%


\defineTitle{18}{Unifikation}{28. Juni 2021}

\begin{document}

\maketitle

\begin{frame}\frametitle{Resolution für Prädikatenlogik}

Ein konkreter Algorithmus zum logischen Schließen:
\begin{enumerate}[(1)]
\item \alert{Logische Konsequenz auf Unerfüllbarkeit reduzieren}
\item \alert{Formeln in Klauselform umwandeln}
	\begin{itemize}
	\item Formel bereinigen
	\item Negationsnormalform bilden
	\item Pränexform bilden
	\item Skolemform bilden
	\item Konjunktive Normalform bilden
	\end{itemize}
\item \alert{Resolutionsverfahren anwenden}
	\begin{itemize}
	\item \textcolor{devilscss}{Unifikation zum Finden passender Klauseln}
	\item \textcolor{devilscss}{Bilden von Resolventen bis zur Terminierung}
	\end{itemize}
\end{enumerate}

\end{frame}

\begin{frame}\frametitle{Konjunktive Normalform in Prädikatenlogik}

\defbox{Eine Formel ist in \redalert{konjunktiver Normalform} (\alert{KNF}) wenn sie 
in Pränexform $\quantor x_1.\cdots\quantor x_n.F$ ist, wobei $F$
eine Konjunktion von Disjunktionen von Literalen ist, welche keine Quantoren enthält.
Das heißt $F$ hat die Form:\\[1ex]
%
\narrowcentering{$(L_{1,1}\vee \ldots\vee L_{1,m_1})\wedge(L_{2,1}\vee \ldots\vee L_{2,m_2})\wedge\ldots
\wedge(L_{n,1}\vee \ldots\vee L_{n,m_n}) $}\\[1ex]
% \narrowcentering{$\bigwedge_{i=1}^n \bigvee_{j=1}^{m_i} L_{i,j}$}\\[1ex]
%
wobei $L_{i,j}$ Literale sind. \redalert{Klauseln} sind Disjunktionen von Literalen.
}\medskip

Wir stellen die KNF in der Prädikatenlogik wie folgt her:
\begin{enumerate}[(1)]
\item Formel bereinigen
\item Bilden der Negationsnormalform
\item Bilden der Pränexform
\item Skolemisieren
\item Erschöpfende Anwendung der folgenden Ersetzung auf Teilformeln im quantorenfreien Teil der Formel:
\begin{align*}
F\vee(G\wedge H) &\mapsto (F\vee G)\wedge (F\vee H)
\end{align*}
\end{enumerate}
\end{frame}

\newcommand{\hi}[2]{\textcolor{white}{\textcolor<#1>{purple}{\underline{\textcolor{HKS41K100}{#2}}}}}
\begin{frame}\frametitle{Beispiel: Konjunktive Normalform}

\footnotesize
\[\begin{array}{rl}
& \forall x_1,x_2,x_3,x_4,x_5.\\
& \Big(\hi{1-}{\big((W(x_1)\wedge\neg L(x_1))\vee (L(x_1)\wedge\neg W(x_1))\big)}\\
& {}\wedge{}  \big(\neg W(x_2)\vee(W(x_3) \vee L(x_4))\big)\\
& {}\wedge{}  \big(\neg L(x_5)\vee(\neg W(f_6(x_1,x_2,x_3,x_4,x_5)) \wedge \neg L(f_7(x_1,x_2,x_3,x_4,x_5)))\big)\Big)\\
{}\equiv{} & \forall x_1,x_2,x_3,x_4,x_5.\\
& \Big((W(x_1)\vee L(x_1))\wedge(\neg L(x_1)\vee L(x_1))\\
& {}\wedge (W(x_1)\vee \neg W(x_1))\wedge(\neg L(x_1)\vee \neg W(x_1))\big)\\
& {}\wedge{}  \big(\neg W(x_2)\vee(W(x_3) \vee L(x_4))\big)\\
& {}\wedge{}  \hi{1-}{\big(\neg L(x_5)\vee(\neg W(f_6(x_1,x_2,x_3,x_4,x_5)) \wedge \neg L(f_7(x_1,x_2,x_3,x_4,x_5)))\big)}\Big)\\
{}\equiv{} & \forall x_1,x_2,x_3,x_4,x_5.\\
& \Big((W(x_1)\vee L(x_1))\wedge(\neg L(x_1)\vee L(x_1))\\
& {}\wedge (W(x_1)\vee \neg W(x_1))\wedge(\neg L(x_1)\vee \neg W(x_1))\big)\\
& {}\wedge{}  \big(\neg W(x_2)\vee(W(x_3) \vee L(x_4))\big)\\
& {}\wedge{}  (\neg L(x_5)\vee\neg W(f_6(x_1,x_2,x_3,x_4,x_5))) \wedge (\neg L(x_5)\vee\neg L(f_7(x_1,x_2,x_3,x_4,x_5)))\Big)
\end{array}
\]

\end{frame}

\begin{frame}\frametitle{Klauselform}

Die \redalert{Klauselform} ist eine vereinfachte Schreibweise der KNF:
\begin{itemize}
\item Allquantoren werden weggelassen
\item Klauseln werden als Mengen von Literalen geschrieben
\item Konjunktionen von Klauseln werden als Mengen von Mengen von Literalen geschrieben
\end{itemize}

\examplebox{\emph{Beispiel:} Unser Beispiel kann damit wie folgt geschrieben werden:\\[-2.5ex]
%
\[\begin{array}{rl}
\{& \{W(x_1), L(x_1)\},\\
& \{\neg L(x_1), L(x_1)\},\\
& \{W(x_1), \neg W(x_1)\},\\
& \{\neg L(x_1), \neg W(x_1)\},\\
& \{\neg W(x_2), W(x_3), L(x_4)\},\\
& \{\neg L(x_5),\neg W(f_6(x_1,x_2,x_3,x_4,x_5))\},\\
& \{\neg L(x_5),\neg L(f_7(x_1,x_2,x_3,x_4,x_5))\}\quad\}
\end{array}
\]
}

\end{frame}

\sectionSlide{Unifikation}

\begin{frame}\frametitle{Idee der Resolution}

Aussagenlogische Resolution verallgemeinert die Verkettung von Implikationen. Beispiel:\bigskip

\[
\frac{
\begin{array}{r}
A \to B\\
{\color{devilscss}\{\neg A, B\}}
\end{array}
\qquad
\begin{array}{r}
B \to C\\
{\color{devilscss}\{\neg B, C\}}
\end{array}
}{
\begin{array}{r}
{\color{devilscss}\{\neg A, C\}}\\
A \to C
\end{array}
}
\]

\pause Dies ist auch in der Prädikatenlogik ein gültiger Schluss:

\[
\frac{
\begin{array}{c}
\forall x.(A(x) \to B(x))\\
{\color{devilscss}\{\neg A(x), B(x)\}}
\end{array}
\qquad
\begin{array}{c}
\forall x.(B(x) \to C(x))\\
{\color{devilscss}\{\neg B(x), C(x)\}}
\end{array}
}{
\begin{array}{c}
{\color{devilscss}\{\neg A(x), C(x)\}}\\
\forall x.(A(x) \to C(x))
\end{array}
}
\]

\end{frame}

\begin{frame}\frametitle{Terme in Klauselform}

Allerdings ist es nicht immer so einfach.\medskip

Schon die alten Griechen wussten:
% 
\[ \frac{
\forall x.(\textsf{Mensch}(x)\to\textsf{Sterblich}(x))\qquad \textsf{Mensch}(\textsf{sokrates})}
{\textsf{Sterblich}(\textsf{sokrates})}
\]

\pause Aber die entsprechenden Klauseln sind:
% 
\[  \{\neg\textsf{Mensch}(x),\textsf{Sterblich}(x)\}\qquad \{\textsf{Mensch}(\textsf{sokrates})\} \]
% 
\emph{Problem:} Wie kann man Resolution über unterschiedliche Atome $\textsf{Mensch}(x)\neq \textsf{Mensch}(\textsf{sokrates})$ durchführen?
\bigskip\pause

\emph{Lösung:} $x$ ist universell quantifiziert und kann für beliebige Elemente stehen, also auch für Sokrates -- logisch gilt:
%
\footnotesize
\[\begin{array}{c}
\forall x.(\textsf{Mensch}(x)\to\textsf{Sterblich}(x))\\
{\color{devilscss}\{\neg\textsf{Mensch}(x),\textsf{Sterblich}(x)\}}
\end{array}
\models
\begin{array}{c}
\textsf{Mensch}(\textsf{sokrates})\to\textsf{Sterblich}(\textsf{sokrates})\\
{\color{devilscss}\{\neg\textsf{Mensch}(\textsf{sokrates}),\textsf{Sterblich}(\textsf{sokrates})\}}
\end{array}
\]

\end{frame}

\begin{frame}\frametitle{Funktionsterme in Klauseln}

Was passiert mit Funktionen?
%
\[ \frac{
\begin{array}{c}
\forall x.(\textsf{Mensch}(x)\to\exists y.\textsf{hatVater}(x,y))\\
\forall z,v.(\textsf{hatVater}(z,v)\to\textsf{hatKind}(v,z))
\end{array}
}
{
\forall x.(\textsf{Mensch}(x)\to\exists y.\textsf{hatKind}(y,x))
}
\]
%
Entsprechende Klauseln:
\[ \frac{
\begin{array}{c}
\{\neg\textsf{Mensch}(x),\textsf{hatVater}(x,f(x))\}\\
\{\neg\textsf{hatVater}(z,v),\textsf{hatKind}(v,z)\}
\end{array}
}
{
\{\neg\textsf{Mensch}(x),\textsf{hatKind}(f(x),x)\}
}
\]
Passen $\textsf{hatVater}(x,f(x))$ und $\neg\textsf{hatVater}(z,v)$ zusammen?\\\pause
\redalert{Ja}, wenn wir $z$ durch $x$ ersetzten und $v$ durch $f(x)$, denn es gilt:

~\hspace{-1cm}%
\begin{minipage}{8cm}
\footnotesize
\[\begin{array}{c}
\forall z,v.(\textsf{hatVater}(z,v)\to\textsf{hatKind}(v,z))\\
{\color{devilscss}\{\neg\textsf{hatVater}(z,v),\textsf{hatKind}(v,z)}
\end{array}
\models
\begin{array}{c}
\forall x.(\textsf{hatVater}(x,f(x))\to\textsf{hatKind}(f(x),x))\\
{\color{devilscss}\{\neg\textsf{hatVater}(x,f(x)),\textsf{hatKind}(f(x),x)\}}
\end{array}
\]
\end{minipage}

\end{frame}

\begin{frame}\frametitle{Zusammenfassung und Verallgemeinerung}

Für die Anwendung von Resolution benötigt man jeweils zwei gleiche Atome\\ (einmal negiert und einmal nicht-negiert)\medskip

Wir erreichen das in Prädikatenlogik durch folgende Methoden:
\begin{itemize}
\item Wir ersetzten (universell quantifizierte) Variablen durch andere Terme $\leadsto$~\redalert{Substitution}
\item Damit wollen wir erreichen, dass Terme (und letztlich Atome) gleich werden $\leadsto$~\redalert{Unifikation}
\end{itemize}

\end{frame}

\begin{frame}\frametitle{Substitutionen}

\defbox{Eine \redalert{Substitution} ist eine endliche Menge der Form\\[0.5ex]
\narrowcentering{$\{x_1\mapsto t_1,\ldots,x_n\mapsto t_n\},$}\\[1ex]
wobei $x_1,\ldots,x_n\in\Slang{V}$ \alert{paarweise verschiedene} Variablen und $t_1,\ldots,t_n\in\Slang{T}$ beliebige Terme sind.\medskip

Die \redalert{Anwendung einer Substitution} auf einen Ausdruck $A$ (Formel oder Term) führt zu einem Ausdruck $A\{x_1\mapsto t_1,\ldots,x_n\mapsto t_n\}$, der aus $A$ entsteht, wenn man jedes \alert{freie} Vorkommen einer Variablen $x_i$ in $A$ \alert{simultan} durch $t_i$ ersetzt.}

Eine Formel bzw. ein Term $A\sigma$, der durch Anwendung einer Substitution $\sigma$
auf $A$ entsteht, nennt man \redalert{Instanz von $A$ (unter $\sigma$)}\pause

\examplebox{\emph{Beispiele:}\vspace{-7.7ex}

\begin{align*}
p(x,y,z)\{x\mapsto f(y),y\mapsto a\} &= p(f(y),a,z) \\[-0.5ex]&~~~~{\footnotesize\text{(und nicht $p(f(a),a,z)$!) }}\\[-0.5ex]
\big(q(x,y)\to\forall x.r(x,y)\big)\{x\mapsto z,y\mapsto f(x)\} &= \big(q(z,f(x))\to\forall x.r(x,f(x))\big)\\[-0.5ex]
& \hspace{-1cm}{\footnotesize\text{(und nicht $\big(q(z,f(x))\to\forall x.r(z,f(x))\big)$!) }}
\end{align*}
}


\end{frame}

\begin{frame}\frametitle{Komposition von Substitutionen}

Wir können Substitutionen hintereinander ausführen:

\defbox{Für Substitutionen $\sigma=\{x_1\mapsto t_1,\ldots,x_n\mapsto t_n\}$
und $\theta=\{y_1\mapsto s_1,\ldots,y_m\mapsto s_m\}$ 
ist die \redalert{Komposition $\sigma\circ\theta$} die folgende Substitution:
\[ \Big\{x_1\mapsto t_1\theta,\ldots,x_n\mapsto t_n\theta\Big\}\;\;\cup\;\;
\Big\{y_i\mapsto s_i\bigm| y_i\in\{y_1,\ldots,y_m\}\setminus\{x_1,\ldots,x_n\}\Big\}\]
}\medskip\pause

\theobox{\emph{Satz:} Für alle Terme oder Formeln $A$ und Substitutionen $\sigma$ und $\theta$ gilt: $A(\sigma\circ\theta)=(A\sigma)\theta$.}\pause

\emph{Beweis:} Per struktureller Induktion über den Aufbau von Termen.\\[1ex]
\alert{I-Anfang:} Satz gilt für Konstanten $A\in\Slang{C}$ und Variablen $A\in\Slang{V}$ per Definition.\\[1ex] \alert{I-Hypothese:} Satz gilt für Terme $t_1,\ldots,t_\ell$.\\[1ex]
\alert{I-Schritt:} Damit gilt für größere Funktionsterme $f(t_1,\ldots,t_\ell)(\sigma\circ\theta)=
f(t_1(\sigma\circ\theta),\ldots,t_\ell(\sigma\circ\theta))\stackrel{\text{IH}}{=}f((t_1\sigma)\theta,\ldots,(t_\ell\sigma)\theta)=(f(t_1,\ldots,t_\ell)\sigma)\theta$.\bigskip

Beweis für Formeln analog (aber mit mehr Fällen im I-Schritt).
\qed

\end{frame}

\begin{frame}\frametitle{Unifikationsprobleme}

\defbox{Ein \redalert{Unifikationsproblem} ist eine endliche Menge von Gleichungen
der Form \mbox{$G=\{s_1\unieq t_1,\ldots,s_n\unieq t_n\}$}.\medskip

Eine Substitution $\sigma$ ist ein \redalert{Unifikator} für $G$ falls
$s_i\sigma=t_i\sigma$ für alle $i\in\{1,\ldots,n\}$ gilt.
}\pause 

\examplebox{\emph{Beispiele:}

\begin{tabular}{rl}
$\{ f(x)\unieq f(\text{sokrates}) \}$ & \pause hat den Unifikator $\{x\mapsto\text{sokrates}\}$\\\pause 
$\{ x\unieq f(y), y\unieq g(z) \}$ & hat den Unifikator $\{x\mapsto f(g(z)), y\mapsto g(z)\}$\pause ,\\
 & aber auch z.B. $\{x\mapsto f(g(a)), y\mapsto g(a), z\mapsto a\}$\\
 & und $\{x\mapsto f(g(f(b))), y\mapsto g(f(b)), z\mapsto f(b)\}$\\\pause 
$\{f(x)\unieq g(x)\}$ &\pause  hat keinen Unifikator\\\pause 
$\{x\unieq f(x)\}$ &\pause  hat keinen Unifikator\\
\end{tabular}
}

\end{frame}

\begin{frame}\frametitle{Unifikatoren vergleichen}

Das Problem $\{ x\unieq f(y), y\unieq g(z) \}$ hat viele Unifikatoren -- gibt es einen "`besten"'?\medskip

\defbox{Eine Substitution $\sigma$ ist \redalert{allgemeiner} als eine Substitution
$\theta$, in Symbolen \redalert{$\sigma\preceq\theta$}, wenn es eine Substitution
$\lambda$ gibt, so dass $\sigma\circ\lambda =\theta$.\medskip

Der \redalert{allgemeinste Unifikator} für ein Unifikationsproblem $G$ ist ein Unifikator $\sigma$ für $G$, so dass $\sigma\preceq\theta$ für alle Unifikatoren $\theta$ für $G$.}

{\footnotesize
(Die englische Bezeichnung des allgemeinsten Unifikators ist \alert{most general unifier (mgu)}.)}\medskip\pause

\examplebox{\emph{Beispiel:} Für $G=\{ x\unieq f(y), y\unieq g(z) \}$ ist
$\sigma=\{x\mapsto f(g(z)), y\mapsto g(z)\}$ ein allgemeinster Unifikator.
Dagegen ist $\theta=\{x\mapsto f(g(f(b))), y\mapsto g(f(b)),z\mapsto f(b)\}$ ein Unifikator, der nicht allgemeinst ist, denn es gilt $\sigma\{z\mapsto f(b)\}=\theta$.}

\end{frame}

\begin{frame}\frametitle{Eindeutigkeit des mgu}

\theobox{\emph{Satz:} Die allgemeinsten Unifikatoren von $G$ sind bis auf Umbenennung von Variablen identisch.}

(ohne Beweis)\medskip\pause

\examplebox{\emph{Beispiel:} Das Problem $\{f(x)\unieq f(z)\}$ hat die allgemeinsten Unifikatoren
$\{x\mapsto z\}$ und $\{z\mapsto x\}$, aber auch z.B. $\{x\mapsto v,z\mapsto v\}$ ($v\in\Slang{V}$). Dagegen ist $\{x\mapsto a,z\mapsto a\}$ mit $a\in\Slang{C}$ ein nicht-allgemeinster Unifikator.}\bigskip\pause

\theobox{\emph{Satz:} Wenn ein Unifikationsproblem lösbar ist (d.h., einen Unifikator hat), dann hat es auch einen allgemeinsten Unifikator.}

Wir zeigen das durch Angabe eines Algorithmus für die mgu-Berechnung.

\end{frame}

\begin{frame}\label{slide_solvedunivsubst}\frametitle{Gelöste Unifikationsprobleme}

\defbox{Ein Unifikationsproblem $G=\{x_1\unieq t_1,\ldots,x_n\unieq t_n\}$ ist in \redalert{gelöster Form}, wenn $x_1,\ldots, x_n$ paarweise verschiedene Variablen sind, die nicht in den Termen $t_1,\ldots,t_n$ vorkommen.\medskip

In diesem Fall definieren wir eine Substitution $\sigma_G\defeq\{x_1\mapsto t_1,\ldots,x_n\mapsto t_n\}$.}\smallskip\pause

\theobox{\emph{Satz:} Wenn $G$ ein Unifikationsproblem in gelöster Form ist, dann ist $\sigma_G$ ein allgemeinster Unifikator für $G$.}

\pause\emph{Beweis:} Sei $G=\{x_1\unieq t_1,\ldots,x_n\unieq t_n\}$. Dann ist
$x_i\sigma_G=t_i=t_i\sigma_G$, wobei die zweite Gleichheit gilt, da $t_i$ keine Variable aus $x_1, \ldots,x_n $ enthält. Also ist $\sigma_G$ Unifikator für $G$.
\smallskip\pause

Für einen beliebigen Unifikator $\theta$ von $G$ gilt: $\theta=\sigma_G\circ\theta$:
\begin{itemize}
\item Für $x_i\in\{x_1,\ldots,x_n\}$ ist $x_i\theta=t_i\theta = (x_i\sigma_G)\theta$
\item Für $y\notin\{x_1,\ldots,x_n\}$ ist $y\theta=y\sigma_G\theta$
\end{itemize}
Also ist $\sigma_G$ ein allgemeinster Unifikator.\qed

\end{frame}

\begin{frame}\frametitle{Lösen von Unifikationsproblemen}

Wir bringen Unifikationsprobleme schrittweise in gelöste Form:

\anybox{strongyellow}{
\redalert{Unifikationsalgorithmus}\\[1ex]
\emph{Eingabe:} Unifikationsproblem $G$\\
\emph{Ausgabe:} allgemeinster Unifikator für $G$, oder "`nicht unifizierbar"'\\[1ex]
%
Wende die folgenden Umformungsregeln auf $G$ an, bis keine Regel mehr zu einer Änderung führt:
\begin{itemize}
\item \alert{Löschen:} $\{t\unieq t\}\cup G' \leadsto G'$
\item \alert{Zerlegung:} $\{f(s_1,\ldots,s_n)\unieq f(u_1,\ldots,u_n)\}\cup G' \leadsto \{s_1\unieq u_1,\ldots,s_n\unieq u_n\}\cup G'$
\item \alert{Orientierung:} $\{t\unieq x\}\cup G' \leadsto \{x\unieq t\}\cup G'$ falls $x\in\Slang{V}$ und $t\notin\Slang{V}$
\item \alert{Eliminierung:} $\{x\unieq t\}\cup G' \leadsto \{x\unieq t\}\cup G'\{x\mapsto t\}$ falls $x\in\Slang{V}$ nicht in $t$ vorkommt 
\end{itemize}
Wenn $G$ dann in gelöster Form ist, dann gib $\sigma_G$ aus.\\
Andernfalls gib aus "`nicht unifizierbar"'.
}

\end{frame}

\begin{frame}\frametitle{Beispiel}

\[
\begin{array}{rl}
\{\hi{2-}{x\unieq f(a)}, g(x,x)\unieq g(x,y)\} & \pause\text{Eliminierung}\\\pause
\{x\unieq f(a), \hi{4-}{g(f(a),f(a))\unieq g(f(a),y)}\} & \pause\text{Zerlegung}\\\pause
\{x\unieq f(a), \hi{6-}{f(a)\unieq f(a)}, f(a)\unieq y\} & \pause\text{Löschen}\\\pause
\{x\unieq f(a), \hi{8-}{f(a)\unieq y}\} & \pause\text{Orientierung}\\\pause
\{x\unieq f(a), y\unieq f(a)\} & \pause\text{gelöste Form}
\end{array}
\]
\bigskip

Die zugehörige Substitution $\{x\mapsto f(a), y\mapsto f(a)\}$ ist Unifikator des
ursprünglichen Unifikationsproblems

\end{frame}

\begin{frame}[t]\frametitle{Korrektheit des Unifikationsalgorithmus (1)}

\theobox{\emph{Satz:} Der Unifikationsalgorithmus berechnet für jedes Unifikationsproblem
einen allgemeinsten Unifikator, falls es einen Unifikator gibt, und liefert "`nicht unifizierbar"' wenn es keinen gibt.}

\emph{Beweis:} Wir zeigen nacheinander drei Aussagen, aus denen die Behauptung folgt.\medskip

Bei der Eingabe beliebiger Unifikationsprobleme $G$ gilt:
\begin{enumerate}[(1)]
\item Wenn der Algorithmus eine Substitution $\sigma$ ausgibt, dann ist $\sigma$ ein allgemeinster Unifikator für $G$
\item Wenn der Algorithmus "`nicht unifizierbar"' ausgibt, dann hat $G$ keinen Unifikator
\item Der Algorithmus terminiert, d.h. erzeugt nach endlich vielen Schritten eine der beiden Ausgaben
\end{enumerate}

\end{frame}

\begin{frame}[t]\frametitle{Korrektheit des Unifikationsalgorithmus (2)}

% \theobox{Satz: Der Unifikationsalgorithmus berechnet für jedes Unifikationsproblem
% einen allgemeinsten Unifikator, falls es einen Unifikator gibt, und liefert "`nicht unifizierbar"' wenn es keinen gibt.}

\emph{Beweis (Fortsetz.):} Als erstes zeigen wir eine Hilfsaussage ($\ddagger$):\\[1ex]
Wenn ein Unifikationsproblem $G_1$ in einem Ersetzungsschritt in ein Problem $G_2$
umgewandelt werden kann, dann haben $G_1$ und $G_2$ die gleichen Unifikatoren.
\bigskip\pause

Für \alert{Löschen}, \alert{Orientierung} und \alert{Zerlegung} ist das leicht zu sehen.\medskip\pause

Für \alert{Eliminierung} betrachten wir $G_1=\{x\unieq t\}\cup G'$ und die Substitution
$\sigma=\{x\mapsto t\}$. Es ist also $G_2=\{x\unieq t\}\cup G'\sigma$.\medskip\pause

\begin{tabular}{rl}
	 & $\theta$ Unifikator für $G_1$ \pause\\[-0.5ex]
gdw. & $x\theta=t\theta$ und $\theta$ Unifikator für $G'$ \pause\\[-0.5ex]
gdw. & $x\theta=t\theta$ und $\sigma\circ\theta$ Unifikator für $G'$ \\[-1ex]
	& \emph{Begründung:} $\sigma=\sigma_{\{x\unieq t\}}$ und daher $\theta=\sigma\circ\theta$\\[-1ex]
	& für alle Unifikatoren $\theta$ von $\{x\unieq t\}$ (gezeigt auf Folie~\ref{slide_solvedunivsubst})\pause\\[-0.5ex]
gdw. & $x\theta=t\theta$ und $\theta$ Unifikator für $G'\sigma$ \pause\\[-0.5ex]
gdw. & $\theta$ Unifikator für $\{x\unieq t\}\cup G'\sigma=G_2$
\end{tabular}

\end{frame}

\begin{frame}[t]\frametitle{Korrektheit des Unifikationsalgorithmus (3)}

% \theobox{Satz: Der Unifikationsalgorithmus berechnet für jedes Unifikationsproblem
% einen allgemeinsten Unifikator, falls es einen Unifikator gibt, und liefert "`nicht unifizierbar"' wenn es keinen gibt.}

\emph{Beweis (Fortsetz.):} (1) Wenn der Algorithmus eine Substitution $\sigma$ ausgibt, dann ist $\sigma$ ein allgemeinster Unifikator für $G$.
\smallskip\pause

\begin{itemize}
\item Gemäß Hilfsaussage ($\ddagger$) erhält jeder Umformungsschritt die Unifikatoren und damit auch die allgemeinsten Unifikatoren\pause
\item Per Induktion gilt also: Jedes Unifikationsproblem, welches der Algorithmus erzeugt, hat den gleichen allgemeinsten Unifikator wie $G$\pause
\item Wenn der Algorithmus einen Unifikator ausgibt, dann ist dies ein allgemeinster Unifikator für eine gelöste Form (Satz auf Folie~\ref{slide_solvedunivsubst})
\end{itemize}
Damit folgt die Behauptung.

\end{frame}

\begin{frame}[t]\frametitle{Korrektheit des Unifikationsalgorithmus (4)}

% \theobox{Satz: Der Unifikationsalgorithmus berechnet für jedes Unifikationsproblem
% einen allgemeinsten Unifikator, falls es einen Unifikator gibt, und liefert "`nicht unifizierbar"' wenn es keinen gibt.}

\emph{Beweis (Fortsetz.):} (2) Wenn der Algorithmus "`nicht unifizierbar"' ausgibt, dann hat $G$ keinen Unifikator.
\smallskip\pause

\emph{Beobachtung:} In diesem Fall erzeugt der Algorithmus ein Problem $G'$, das eine der folgenden Gleichungen enthält:
\begin{enumerate}[(a)]
\item $x\unieq t$, wobei die Variable $x\in\Slang{V}$ in $t$ vorkommt
\item $f(\cdots)\unieq g(\cdots)$ mit $f\neq g$
\end{enumerate}
\pause\emph{Begründung:} $G'$ muss eine Gleichung $s\unieq u$ enthalten, die nicht in gelöster Form ist.
\begin{itemize}
\item Fall 1: $s\unieq u$ hat auf einer Seite eine Variable. Dann hat sie die Form (a), da sonst Orientierung oder Löschen möglich wäre.
\item Fall 2: $s\unieq u$ hat auf keiner Seite eine Variable. Dann hat sie die Form (b), da sonst Zerlegung möglich wäre
\end{itemize}
\pause Aus der Beobachtung folgt (2), da Gleichungen (a) und (b) keinen Unifikator haben.
Laut ($\ddagger$) hat damit auch $G$ keinen Unifikator.


\end{frame}

\begin{frame}[t]\frametitle{Korrektheit des Unifikationsalgorithmus (5)}

% \theobox{Satz: Der Unifikationsalgorithmus berechnet für jedes Unifikationsproblem
% einen allgemeinsten Unifikator, falls es einen Unifikator gibt, und liefert "`nicht unifizierbar"' wenn es keinen gibt.}

\emph{Beweis (Fortsetz.):} (3) Der Algorithmus terminiert, d.h. erzeugt nach endlich vielen Schritten eine der beiden Ausgaben.
\smallskip

\emph{Ansatz:} Wir ordnen Unifikationsprobleme so an, dass das aktuelle Problem in jedem Schritt kleiner wird, und argumentieren, dass dies nicht ewig so weitergehen kann.\medskip\pause

\anybox{strongyellow}{
Dazu definieren wir für jedes Problem $G'$ ein Tripel natürlicher Zahlen $\kappa(G')=\tuple{v,g,r}$:
\begin{itemize}
\item $v$ ist die Zahl der nicht gelösten \redalert{V}ariablen in $G'$\\[1ex]
Eine Variable $x$ ist gelöst, wenn sie in $G'$ nur in einer Gleichung $x\unieq t$
vorkommt und dabei nicht in $t$ enthalten ist
\item $g$ ist die \redalert{G}esamtzahl der Vorkommen von Funktionssymbolen, Konstanten und Variablen in $G'$
\item $r$ ist die Zahl der Gleichungen $s\unieq x\in G'$ mit Variable $x\in\Slang{V}$ auf der \redalert{r}echten Seite
\end{itemize}}

\end{frame}

\begin{frame}[t]\frametitle{Korrektheit des Unifikationsalgorithmus (6)}

% \theobox{Satz: Der Unifikationsalgorithmus berechnet für jedes Unifikationsproblem
% einen allgemeinsten Unifikator, falls es einen Unifikator gibt, und liefert "`nicht unifizierbar"' wenn es keinen gibt.}

\emph{Beweis (Fortsetz.):} (3) Der Algorithmus terminiert, d.h. erzeugt nach endlich vielen Schritten eine der beiden Ausgaben.
\smallskip

Wir ordnen Unifikationsprobleme $G'$ \alert{lexikographisch} bezüglich der Tripel $\kappa(G')$.\smallskip

Für $G_1$ und $G_2$ mit $\kappa(G_i)=\tuple{v_i,g_i,r_i}$ gilt $G_1\succ G_2$, falls:
\begin{itemize}
\item $v_1>v_2$ oder
\item $v_1=v_2$ und $g_1>g_2$ oder
\item $v_1=v_2$ und $g_1=g_2$ und $r_1>r_2$.
\end{itemize}\pause

\examplebox{\emph{Beispiel:} $\tuple{4,2,1}\succ\tuple{3,42,23}\succ\tuple{3,42,22}\succ\tuple{3,41,1000}$.}\pause

In dieser Ordnung gibt es keine unendlichen absteigenden Ketten immer kleiner werdender Unifikationsprobleme: Die Ordnung ist \redalert{wohlfundiert}. (ohne Beweis)

\end{frame}

\begin{frame}[t]\frametitle{Korrektheit des Unifikationsalgorithmus (7)}

% \theobox{Satz: Der Unifikationsalgorithmus berechnet für jedes Unifikationsproblem
% einen allgemeinsten Unifikator, falls es einen Unifikator gibt, und liefert "`nicht unifizierbar"' wenn es keinen gibt.}

\emph{Beweis (Fortsetz.):} (3) Der Algorithmus terminiert, d.h. erzeugt nach endlich vielen Schritten eine der beiden Ausgaben.
\smallskip

Die Behauptung folgt nun, da jede Regelanwendung zu einem $\succ$-kleineren Unifikationsproblem führt:\pause

\anybox{strongyellow}{\vspace{-2ex}
\begin{itemize}
\item $v$: Zahl der nicht gelösten Variablen
\item $g$: Gesamtzahl von Termen
\item $r$: Zahl der Gleichungen $s\unieq x\in G'$ mit Variable $x\in\Slang{V}$
\end{itemize}}

Der Effekt der Regeln ist dabei wie folgt:

\begin{tabular}{@{}rccc}
	& $v$ & $g$ & $r$ \\[-0.5ex]
\alert{Löschen} & $\geq$ & $>$ & \\[-0.5ex]
\alert{Zerlegen} & $\geq$ & $>$ & \\[-0.5ex]
\alert{Orientieren} & $\geq$ & $=$ & $>$ \\[-0.5ex]
\alert{Eliminierung} & $>$ & &
\end{tabular}\hspace{5mm}
\begin{minipage}{5.0cm}
~\\~\\~\\
Da es keine unendlichen Ketten immer kleinerer Probleme geben kann, muss der Algorithmus irgendwann terminieren.\qed
\end{minipage}

\end{frame}

\begin{frame}\frametitle{Beispiel: Terminierungsordnung}

\[
\begin{array}{rlccc}
&& v & g & r\\
\{\hi{1-}{x\unieq f(a)}, g(x,x)\unieq g(x,y)\} & \text{Eliminierung} & 2 & 9 & 0 \\
\{x\unieq f(a), \hi{1-}{g(f(a),f(a))\unieq g(f(a),y)}\} & \text{Zerlegung} & 1 & 12 & 0\\
\{x\unieq f(a), \hi{1-}{f(a)\unieq f(a)}, f(a)\unieq y\} & \text{Löschen} & 1 & 10 & 1\\
\{x\unieq f(a), \hi{1-}{f(a)\unieq y}\} & \text{Orientierung} & 1 & 6 & 1\\
\{x\unieq f(a), y\unieq f(a)\} & \text{gelöste Form} & 0 & 6 & 0
\end{array}
\]\bigskip

Es gilt: $\tuple{2,9,0}\succ\tuple{1,12,0}\succ\tuple{1,10,1}\succ\tuple{1,6,1}\succ\tuple{0,6,0}$

\end{frame}

% \begin{frame}\frametitle{Unifikation von Atomen}
% 
% \defbox{Ein \redalert{Unifikator} für eine Menge $\mathcal{A}=\{A_1,\ldots,A_n\}$ von prädikatenlogischen Atomen ist eine Substitution $\theta$ mit $A_1\theta=A_2\theta=\ldots=A_n\theta$.
% }\pause
% 
% \emph{Beobachtungen:}
% \begin{itemize}
% \item Eine Menge von Atomen $\mathcal{A}$ ist nur dann unifizierbar, wenn alle Atome das gleiche Prädikat verwenden, d.h. wenn es ein $\ell$-stelliges Prädikatensymbol $p$ gibt, so dass $A_i=p(t_{i,1},\ldots,t_{i,\ell})$ für alle $i\in\{1,\ldots,n\}$.
% \item Dann ist $\sigma$ genau dann ein Unifikator für $\mathcal{A}$ wenn
% $\sigma$ Unifikator für das folgende Unifikationsproblem $G_{\mathcal{A}}$ ist: 
% \[ \{ t_{1,1}\unieq t_{2,1},\ldots,t_{n-1,1}\unieq t_{n,1},\ldots,t_{1,\ell}\unieq t_{2,\ell},\ldots,t_{n-1,\ell}\unieq t_{n,\ell} \} \]
% \item Insbesondere ist der allgemeinste Unifikator für $G_{\mathcal{A}}$ auch der allgeeinste Unifikator für $\mathcal{A}$
% \end{itemize}
% 
% \end{frame}
% 
% \begin{frame}\frametitle{Die Resolutionsregel}
% 
% \defbox{Die \redalert{Resolvente} von zwei Klauseln der Form\\[1ex]
% \narrowcentering{$K_1=\{A_1,\ldots,A_n,L_1,\ldots,L_k\}$ und $K_2=\{\neg A'_1,\ldots,\neg A'_m,L'_1,\ldots,L'_\ell\}$,}\\[1ex]
% für die $\sigma$ der allgemeinste Unifikator der Menge $\{A_1,\ldots,A_n,A'_1,\ldots,A'_m\}$
% ist und $L_i,L'_j$ beliebige Literale sind, ist die Klausel
% $\{L_1\sigma,\ldots,L_k\sigma,L'_1\sigma,\ldots,L'_\ell\sigma\}$.
% }\medskip\pause
% 
% \examplebox{Beispiel: Die Klauseln $K_1=\{\neg\textsf{Mensch}(x),\textsf{hatVater}(x,f(x))\}$ und $K_2=\{\neg\textsf{hatVater}(z,v),\textsf{hatKind}(v,z)\}$ können resolviert werden. Ein allgemeinster Unifikator von $\{\textsf{hatVater}(x,f(x)),\textsf{hatVater}(z,v)\}$
% ist $\sigma=\{z\mapsto x,v\mapsto f(x)\}$.
% Die entsprechende Resolvente von $K_1$ und $K_2$ ist $\{\neg\textsf{Mensch}(x),\textsf{hatKind}(f(x),x)\}$.
% }
% 
% \end{frame}
% 
% \begin{frame}\frametitle{Varianten von Klauseln}
% 
% Es kann störend sein, wenn unterschiedliche Klauseln die selbe Variable verwenden.
% \medskip
% 
% \emph{Beispiel:} Für $\forall x.(\textsf{Mensch}(x)\to\exists y.(\textsf{hatVater}(x,y)\wedge\textsf{Mensch}(y)))$ erhalten wir die Klauselform
% %
% \[ \{\{\neg \textsf{Mensch}(x),\textsf{hatVater}(x,f(x))\},\{\neg \textsf{Mensch}(x),\textsf{Mensch}(f(x))\}\} \]
% %
% Die Klauseln sind nicht resolvierbar, weil $\{\neg \textsf{Mensch}(x),\textsf{Mensch}(f(x))\}$ keinen Unifikator hat.
% \bigskip\pause
% 
% Da jede Klausel für sich eine allquantifizierte Formel kodiert ($\forall$ distribuiert über $\wedge$) darf man Variablen aber umbenennen. Wir können z.B. die folgende \redalert{Variante} der zweiten Klausel bilden:
% \[ \{\neg \textsf{Mensch}(y),\textsf{Mensch}(f(y))\} \]
% $\{\neg \textsf{Mensch}(x),\textsf{Mensch}(f(y))\}$ hat den einen Unifikator, so dass wir z.B. die folgende Resolvente erhalten:
% \[\{\neg \textsf{Mensch}(y),\textsf{hatVater}(f(y),f(f(y)))\}\]
% 
% \end{frame}

% \begin{frame}\frametitle{Resolutionsalgorithmus}
% 
% 
% 
% \end{frame}


\begin{frame}\frametitle{Zusammenfassung und Ausblick}

Bei der prädikatenlogischen Resolution müssen Atome unifiziert werden\bigskip

Unifikation von Termen findet mit Hilfe von Substitutionen statt
\bigskip

Der Unifikationsalgorithmus erlaubt uns, allgemeinste Unifikatoren zu berechnen, falls
diese existieren
\bigskip

\anybox{yellow}{
Was erwartet uns als nächstes?
\begin{itemize}
\item Der Resolutionsalgorithmus und seine Korrektheit
\item Herbrand, genialer Mathematiker aber unglücklicher Bergsteiger
\item Logik über endlichen Modellen und ihre praktische Anwendung
\end{itemize}
}

\end{frame}




% \begin{frame}[t]\frametitle{Literatur und Bildrechte}
% 
% \alert{Literatur}\bigskip
% 
% \begin{itemize}
% \item Thoralf A. Skolem: \emph{Logisch-kombinatorische Untersuchungen über die Erfüllbarkeit und Beweisbarkeit mathematischen Sätze nebst einem Theoreme über dichte Mengen.} Videnskapsselskapets Skrifter. I. Mat.-naturv Klasse, 1920, No. 4. Kristiania 1920.
% \item Hao Wang: \emph{Skolem and Gödel.} Nordic Journal of Philosophical Logic, Vol. 1, No. 2, pp. 119–132. \url{http://www.hf.uio.no/ifikk/forskning/publikasjoner/tidsskrifter/njpl/vol1no2/skogod.pdf}
% \end{itemize}
% 
% \bigskip
% 
% \alert{Bildrechte}\bigskip
% 
% Folie \ref{frame_skolem}: Fotografie, 1930er (heute Oslo Museum, Inventarnummer OB.F06426c), gemeinfrei
% 
% \end{frame}


\end{document}
