\documentclass[aspectratio=1610,onlymath]{beamer}
% \documentclass[aspectratio=1610,onlymath,handout]{beamer}

% Macros used by all lectures, but not necessarily by excercises

% Common notation

\usepackage{amsmath,amssymb,amsfonts}
\usepackage{xspace}

\newcommand{\lectureurl}{https://iccl.inf.tu-dresden.de/web/TheoLog2024}

\DeclareMathAlphabet{\mathsc}{OT1}{cmr}{m}{sc} % Let's have \mathsc since the slide style has no working \textsc

% Dual of "phantom": make a text that is visible but intangible
\newcommand{\ghost}[1]{\raisebox{0pt}[0pt][0pt]{\makebox[0pt][l]{#1}}}

\newcommand{\tuple}[1]{\langle{#1}\rangle}
\newcommand{\defeq}{\mathrel{:=}}

%%% Annotation %%%

\usepackage{color}
\newcommand{\todo}[1]{{\tiny\color{red}\textbf{TODO: #1}}}


\newcommand{\quantor}{\mathord{\reflectbox{$\text{\sf{Q}}$}}} % the generic quantor
\newcommand{\unieq}{\stackrel{.}{=}} % equality sign for unification problems


%%% Old macros below; move when needed

\newcommand{\blank}{\text{\textvisiblespace}} % empty tape cell for TM

% table syntax
\newcommand{\dom}{\textbf{dom}}
\newcommand{\adom}{\textbf{adom}}
\newcommand{\dbconst}[1]{\texttt{"#1"}}
\newcommand{\pred}[1]{\textsf{#1}}
\newcommand{\foquery}[2]{#2[#1]}
\newcommand{\ground}[1]{\textsf{ground}(#1)}
% \newcommand{\foquery}[2]{\{#1\mid #2\}} %% Notation as used in Alice Book
% \newcommand{\foquery}[2]{\tuple{#1\mid #2}}

% logic syntax
\newcommand{\Inter}{\mathcal{I}} %used to denote an interpretation
\newcommand{\Jnter}{\mathcal{J}} %used to denote another interpretation
\newcommand{\Knter}{\mathcal{K}} %used to denote yet another interpretation
\newcommand{\Zuweisung}{\mathcal{Z}} %used to denote a variable assignment

% query languages
\newcommand{\qlang}[1]{{\sf #1}} % Font for query languages
\newcommand{\qmaps}[1]{\textbf{QM}({\sf #1})} % Set of query mappings for a query language

%%% Complexities %%%

\hyphenation{Exp-Time} % prevent "Ex-PTime" (see, e.g. Tobies'01, Glimm'07 ;-)
\hyphenation{NExp-Time} % better that than something else

% \newcommand{\complclass}[1]{{\sc #1}\xspace} % font for complexity classes
\newcommand{\complclass}[1]{\ensuremath{\mathsc{#1}}\xspace} % font for complexity classes

\newcommand{\ACzero}{\complclass{AC$_0$}}
\newcommand{\LogSpace}{\complclass{L}}
\newcommand{\NLogSpace}{\complclass{NL}}
\newcommand{\PTime}{\complclass{P}}
\newcommand{\NP}{\complclass{NP}}
\newcommand{\coNP}{\complclass{coNP}}
\newcommand{\PH}{\complclass{PH}}
\newcommand{\PSpace}{\complclass{PSpace}}
\newcommand{\NPSpace}{\complclass{NPSpace}}
\newcommand{\ExpTime}{\complclass{ExpTime}}
\newcommand{\NExpTime}{\complclass{NExpTime}}
\newcommand{\ExpSpace}{\complclass{ExpSpace}}
\newcommand{\TwoExpTime}{\complclass{2ExpTime}}
\newcommand{\NTwoExpTime}{\complclass{N2ExpTime}}
\newcommand{\ThreeExpTime}{\complclass{3ExpTime}}
\newcommand{\kExpTime}[1]{\complclass{#1ExpTime}}
\newcommand{\kExpSpace}[1]{\complclass{#1ExpSpace}}


%%% Style commands

\newcommand{\quoted}[1]{\texttt{"}{#1}\texttt{"}}
\newcommand{\squote}{\texttt{"}} % straight quote
\newcommand{\Sterm}[1]{\ensuremath{\mathtt{\textcolor{purple}{#1}}}}    % letters in alphabets
\newcommand{\Snterm}[1]{\textsf{\textcolor{darkblue}{#1}}} % nonterminal symbols
\newcommand{\Sntermsub}[2]{\ensuremath{\Snterm{#1}_{\Snterm{#2}}}} % nonterminal symbols
\newcommand{\Slang}[1]{\textbf{\textcolor{black}{#1}}}    % languages
\newcommand{\Slangsub}[2]{\ensuremath{\Slang{#1}_{\Slang{#2}}}}    % languages
% Code
\newcommand{\Scode}[1]{\textbf{#1}}    % reserved words in program listings, e.g., "if"
\newcommand{\Scodelit}[1]{\textcolor{purple}{#1}}    % literals in program listings, e.g., strings
\newcommand{\Scomment}[1]{\textcolor{gray}{#1}}    % comment in program listings
% LOOP and WHILE programs
\newcommand{\Svar}[1]{\texttt{#1}} % variable names
\newcommand{\Svsub}[2]{\ensuremath{\Svar{#1}_{\Svar{#2}}}} % variable names
\newcommand{\Sxsub}[1]{\Svsub{x}{#1}} % variable names
\newcommand{\Sseq}{\texttt{;}}
\newcommand{\SStartLoop}[1]{\Scode{LOOP}~#1~\Scode{DO}~}
\newcommand{\SEndLoop}{\Scode{END}}
\newcommand{\SStartIf}[1]{\Scode{IF}~#1~\Scode{THEN}~}
\newcommand{\SElse}{\Scode{ELSE}}
\newcommand{\SEndIf}{\Scode{END}}
\newcommand{\Svassign}{\texttt{ := }} % variable assignments
\newcommand{\Svneq}{\texttt{!=}\ensuremath{\,}} % inequality operator
\newcommand{\Splus}{\texttt{ + }} % addition operator
\newcommand{\Sminus}{\texttt{ - }} % subtraction operator
\newcommand{\SStartWhile}[1]{\Scode{WHILE}~#1\Svneq\texttt{\Scodelit{0}}~\Scode{DO}~}
\newcommand{\SEndWhile}{\Scode{END}}

\newcommand{\bbfunc}{\boldsymbol{\Sigma}}

\newcommand{\epstrastar}{\mathrel{\mathord{\stackrel{\epsilon}{\to}}{}^*}} % transitive reflexive closure of epsilon transitions in an epslion-NFA

\newcommand{\narrowcentering}[1]{\mbox{}\hfill#1\hfill\mbox{}}

\newcommand{\Smach}[1]{\ensuremath{\mathcal{#1}}}    % machines

\newcommand{\mytrue}{\Scodelit{1}}
\newcommand{\myfalse}{\Scodelit{0}}
% \newcommand{\emptyClause}{\bot}

\newcommand{\Scomplclass}[1]{{\textsc{#1}}\xspace} % font for complexity classes, used on slides where the "too many alphabets" LaTeX error appears when using the correct sc font :-(
% \newcommand{\complclass}[1]{\ensuremath{\mathsc{#1}}} % font for complexity classes


\DeclareMathOperator{\enc}{\mathsf{enc}}

\let\complclass\Scomplclass % prevent "too many math alphabets" error on some slides

%%% General setup and dependencies:

% \usetheme[ddcfooter,nosectionnum]{tud}
\usetheme[nosectionnum,pagenum,noheader]{tud}
% \usetheme[nosectionnum,pagenum]{tud}
\setbeamertemplate{enumerate items}[default]

% Increase body font size to a sane level:
\let\origframetitle\frametitle
% \renewcommand{\frametitle}[1]{\origframetitle{#1}\normalsize}
\renewcommand{\frametitle}[1]{\origframetitle{#1}\fontsize{10pt}{13.2}\selectfont}
\setbeamerfont{itemize/enumerate subbody}{size=\small} % tud defaults to scriptsize!
\setbeamerfont{itemize/enumerate subsubbody}{size=\small}
% \setbeamerfont{normal text}{size=\small}
% \setbeamerfont{itemize body}{size=\small}

\renewcommand{\emph}[1]{\textbf{#1}}

\def\arraystretch{1.3}% Make tables even less cramped vertically

\usepackage[ngerman]{babel}
\usepackage[utf8]{inputenc}
\usepackage[T1]{fontenc}

%\usepackage{graphicx}
\usepackage[export]{adjustbox} % loads graphicx
\usepackage{import}
\usepackage{stmaryrd}
\usepackage[normalem]{ulem} % sout command
% \usepackage{times}
\usepackage{txfonts}
\usepackage{array}

% \usepackage[perpage]{footmisc} % reset footnote counter on each page -- fails with beamer (footnotes gone)
\usepackage{perpage}  % reset footnote counter on each page
\MakePerPage{footnote}

\usepackage{tikz}
\usetikzlibrary{arrows,positioning,decorations.pathreplacing}
% Inspired by http://www.texample.net/tikz/examples/hand-drawn-lines/
\usetikzlibrary{decorations.pathmorphing}
\pgfdeclaredecoration{penciline}{initial}{
    \state{initial}[width=+\pgfdecoratedinputsegmentremainingdistance,
    auto corner on length=1mm,]{
        \pgfpathcurveto%
        {% From
            \pgfqpoint{\pgfdecoratedinputsegmentremainingdistance}
                      {\pgfdecorationsegmentamplitude}
        }
        {%  Control 1
        \pgfmathrand
        \pgfpointadd{\pgfqpoint{\pgfdecoratedinputsegmentremainingdistance}{0pt}}
                    {\pgfqpoint{-\pgfdecorationsegmentaspect
                     \pgfdecoratedinputsegmentremainingdistance}%
                               {\pgfmathresult\pgfdecorationsegmentamplitude}
                    }
        }
        {%TO
        \pgfpointadd{\pgfpointdecoratedinputsegmentlast}{\pgfpoint{1pt}{1pt}}
        }
    }
    \state{final}{}
}
\tikzset{handdrawn/.style={decorate,decoration=penciline}}
\tikzset{every shadow/.style={fill=none,shadow xshift=0pt,shadow yshift=0pt}}
% \tikzset{module/.append style={top color=\col,bottom color=\col}}

% Use to make Tikz attributes with Beamer overlays
% http://tex.stackexchange.com/a/6155
\tikzset{onslide/.code args={<#1>#2}{%
  \only<#1| handout:0>{\pgfkeysalso{#2}}
}}
\tikzset{onslideprint/.code args={<#1>#2}{%
  \only<#1>{\pgfkeysalso{#2}}
}}

%%% Title -- always set this first

\newcommand{\defineTitle}[3]{
	\newcommand{\lectureindex}{#1}
	\title{Theoretische Informatik und Logik}
	\subtitle{\href{\lectureurl}{#1. Vorlesung: #2}}
	\author{\href{https://iccl.inf.tu-dresden.de/web/Markus_Kr\%C3\%B6tzsch}{Markus Kr\"{o}tzsch}\\[1ex]Professur Wissensbasierte Systeme}
	\date{#3}
	\datecity{TU Dresden}
% 	\institute{CC-By 3.0, sofern keine anderslautenden Bildrechte angegeben sind}
}

%%% Table of contents:

\RequirePackage{ifthen}

\newcommand{\highlight}[2]{%
	\ifthenelse{\equal{#1}{\lectureindex}}{\alert{#2}}{#2}%
}

\def\myspace{-0.7ex}
\newcommand{\printtoc}{
\begin{tabular}{r@{$\quad$}l}
\highlight{1}{1.} & \highlight{1}{Willkommen/Einleitung formale Sprachen}\\[\myspace]
\highlight{2}{2.} & \highlight{2}{Grammatiken und die Chomsky-Hierarchie}\\[\myspace]
\highlight{3}{3.} & \highlight{3}{Endliche Automaten}\\[\myspace]
\highlight{4}{4.} & \highlight{4}{Complexity of FO query answering}\\[\myspace]
\highlight{5}{5.} & \highlight{5}{Conjunctive queries}\\[\myspace]
\highlight{6}{6.} & \highlight{6}{Tree-like conjunctive queries}\\[\myspace]
\highlight{7}{7.} & \highlight{7}{Query optimisation}\\[\myspace]
\highlight{8}{8.} & \highlight{8}{Conjunctive Query Optimisation / First-Order~Expressiveness}\\[\myspace]
\highlight{9}{9.} & \highlight{9}{First-Order~Expressiveness / Introduction to Datalog}\\[\myspace]
\highlight{10}{10.} & \highlight{10}{Expressive Power and Complexity of Datalog}\\[\myspace]
\highlight{11}{11.} & \highlight{11}{Optimisation and Evaluation of Datalog}\\[\myspace]
\highlight{12}{12.} & \highlight{12}{Evaluation of Datalog (2)}\\[\myspace]
\highlight{13}{13.} & \highlight{13}{Graph Databases and Path Queries}\\[\myspace]
\highlight{14}{14.} & \highlight{14}{Outlook: database theory in practice}
\end{tabular}
}

\newcommand{\overviewslide}{%
\begin{frame}\frametitle{Overview}
\printtoc
\medskip

Siehe \href{\lectureurl}{course homepage [$\Rightarrow$ link]} for more information and materials
\end{frame}
}

%%% Colours:
\usepackage{xcolor,colortbl}
\definecolor{redhighlights}{HTML}{FFAA66}
\definecolor{lightblue}{HTML}{55AAFF}
\definecolor{lightred}{HTML}{FF5522}
\definecolor{lightpurple}{HTML}{DD77BB}
\definecolor{lightgreen}{HTML}{55FF55}
\definecolor{lightgray}{HTML}{CCCCCC}
\definecolor{darkred}{HTML}{CC4411}
\definecolor{darkblue}{HTML}{176FC0}%{1133AA}
\definecolor{nightblue}{HTML}{2010A0}%{1133AA}
\definecolor{alert}{HTML}{176FC0}
\definecolor{darkgreen}{HTML}{36AB14}
\definecolor{strongyellow}{HTML}{FFE219}
\definecolor{devilscss}{HTML}{666666}

\newcommand{\redalert}[1]{\textcolor{darkred}{#1}}

%%% Slide layout commands:

\newcommand{\sectionSlide}[1]{
\frame{\begin{center}
\LARGE
#1
\end{center}}
}
\newcommand{\sectionSlideNoHandout}[1]{
\frame<handout:0>{\begin{center}
\LARGE
#1
\end{center}}
}

\newcommand{\mydualbox}[3]{%
 \begin{minipage}[t]{#1}
 \begin{beamerboxesrounded}[upper=block title,lower=block body,shadow=true]%
    {\centering\usebeamerfont*{block title}#2}%
    \raggedright%
    \usebeamerfont{block body}
%     \small
    #3%
  \end{beamerboxesrounded}
  \end{minipage}
}
%
\newcommand{\myheaderbox}[2]{%
 \begin{minipage}[t]{#1}
 \begin{beamerboxesrounded}[upper=block title,lower=block title,shadow=true]%
    {\centering\usebeamerfont*{block title}\rule{0pt}{2.6ex} #2}%
  \end{beamerboxesrounded}
  \end{minipage}
}

\newcommand{\mycontentbox}[2]{%
 \begin{minipage}[t]{#1}%
 \begin{beamerboxesrounded}[upper=block body,lower=block body,shadow=true]%
    {\centering\usebeamerfont*{block body}\rule{0pt}{2.6ex}#2}%
  \end{beamerboxesrounded}
  \end{minipage}
}

\newcommand{\mylcontentbox}[2]{%
 \begin{minipage}[t]{#1}%
 \begin{beamerboxesrounded}[upper=block body,lower=block body,shadow=true]%
    {\flushleft\usebeamerfont*{block body}\rule{0pt}{2.6ex}#2}%
  \end{beamerboxesrounded}
  \end{minipage}
}

% label=180:{\rotatebox{90}{{\footnotesize\textcolor{darkgreen}{Beispiel}}}}
% \hspace{-8mm}\ghost{\raisebox{-7mm}{\rotatebox{90}{{\footnotesize\textcolor{darkgreen}{Beispiel}}}}}\hspace{8mm}
\newcommand{\examplebox}[1]{%
	\begin{tikzpicture}[decoration=penciline, decorate]
		\pgfmathsetseed{1235}
		\node (n1) [decorate,draw=darkgreen, fill=darkgreen!10,thick,align=left,text width=\linewidth, inner ysep=2mm, inner xsep=2mm] at (0,0) {#1};
% 		\node (n2) [align=left,text width=\linewidth,inner sep=0mm] at (n1.92) {{\footnotesize\raisebox{3mm}{\textcolor{darkgreen}{Beispiel}}}};
% 		\node (n2) [decorate,draw=darkgreen, fill=darkgreen!10,thick, align=left,text width=\linewidth,inner sep=2mm] at (n1.90) {{\footnotesize\raisebox{0mm}{\textcolor{darkgreen}{Beispiel}}}};
	\end{tikzpicture}%
}%

\newcommand{\codebox}[1]{%
	\begin{tikzpicture}[decoration=penciline, decorate]
		\pgfmathsetseed{1236}
		\node (n1) [decorate,draw=strongyellow, fill=strongyellow!10,thick,align=left,text width=\linewidth, inner ysep=2mm, inner xsep=2mm] at (0,0) {#1};
	\end{tikzpicture}%
}%

\newcommand{\defbox}[1]{%
	\begin{tikzpicture}[decoration=penciline, decorate]
		\pgfmathsetseed{1237}
		\node (n1) [decorate,draw=darkred, fill=darkred!10,thick,align=left,text width=\linewidth, inner ysep=2mm, inner xsep=2mm] at (0,0) {#1};
	\end{tikzpicture}%
}%

\newcommand{\theobox}[1]{%
	\begin{tikzpicture}[decoration=penciline, decorate]
		\pgfmathsetseed{1240}
		\node (n1) [decorate,draw=darkblue, fill=darkblue!10,thick,align=left,text width=\linewidth, inner ysep=2mm, inner xsep=2mm] at (0,0) {#1};
	\end{tikzpicture}%
}%

\newcommand{\anybox}[2]{%
	\begin{tikzpicture}[decoration=penciline, decorate]
		\pgfmathsetseed{1240}
		\node (n1) [decorate,draw=#1, fill=#1!10,thick,align=left,text width=\linewidth, inner ysep=2mm, inner xsep=2mm] at (0,0) {#2};
	\end{tikzpicture}%
}%


\newsavebox{\mybox}%
\newcommand{\doodlebox}[2]{%
\sbox{\mybox}{#2}%
	\begin{tikzpicture}[decoration=penciline, decorate]
		\pgfmathsetseed{1238}
		\node (n1) [decorate,draw=#1, fill=#1!10,thick,align=left,inner sep=1mm] at (0,0) {\usebox{\mybox}};
	\end{tikzpicture}%
}%


\defineTitle{20}{Resolution (2)}{5. Juli 2021}

\begin{document}

\maketitle

\bgroup
\setbeamercolor{background canvas}{bg=black}
\frame[plain]{\label{frame_herbrand}\begin{center}\color{white}
\includegraphics[height=6cm]{images/Herbrand.jpg}

\LARGE
Jacques Herbrand\medskip

\normalsize
\tt 12.2.1908 -- 27.7.1931
\end{center}}
\egroup

\sectionSlide{Evaluation}

\begin{frame}\frametitle{Vollständigkeit der Resolution}

Resolutionsregel:
\[  \frac{
\{A_1,\ldots,A_n,L_1,\ldots,L_k\}\quad
\{\neg A'_1,\ldots,\neg A'_m,L'_1,\ldots,L'_\ell\}
}{\{L_1\sigma,\ldots,L_k\sigma,L'_1\sigma,\ldots,L'_\ell\sigma\}
}\]
falls $\sigma$ allgemeinster Unifikator von $\{A_1,\ldots,A_n, A'_1,\ldots, A'_m\}$ ist\bigskip

\emph{Algorithmus (Skizze):}
\begin{enumerate}[(1)]
\item Bilde Klauselform
\item Bilde systematisch Resolventen durch Resolution von Varianten bereits abgeleiteter Klauseln
\item Wiederhole (2), bis entweder $\bot$ erzeugt wird ("`unerfüllbar"') oder keine neuen Klauseln mehr entstehen\footnote{Dieser Fall ist eher ungewöhnlich: Meist entstehen bei erfüllbaren Theorien immer mehr neue Klauseln ohne dass das Verfahren terminiert.}
\end{enumerate}

\end{frame}

\begin{frame}\frametitle{Vollständigkeit und Korrektheit}

\theobox{Resolutionssatz: Sei $F$ eine prädikatenlogische Formel und $\mathcal{K}_i$ ($i\geq 0$) die vom Resolutionsalgorithmus ermittelten Klauselmengen. Dann sind die folgenden Aussagen äquivalent:
\begin{itemize}
\item $F$ ist unerfüllbar
\item Es gibt ein $\ell\geq 0$ mit $\bot\in\mathcal{K}_\ell$
\end{itemize}}

\begin{itemize}
\item Korrektheit hatten wir bereits gezeigt
\item Vollständigkeit steht noch aus
\end{itemize}

\end{frame}

\bgroup
\setbeamercolor{background canvas}{bg=black}
\frame[plain]{\label{frame_herbrand_b}\begin{center}\color{white}
\includegraphics[height=6cm]{images/Herbrand.jpg}

\LARGE
Jacques Herbrand\medskip

\normalsize
\tt 12.2.1908 -- 27.7.1931
\end{center}}
\egroup

\begin{frame}\frametitle{Syntax vs. Semantik}

Bei Herbrandinterpretationen kann man semantische Elemente (wie sie in Zuweisungen vorkommen) durch syntaktische Elemente (wie sie in Substitutionen vorkommen) ausdrücken:\medskip

\theobox{Lemma: Für jede Herbrandinterpretation $\Inter$, jede Zuweisung $\Zuweisung$ für $\Inter$, jeden Term $t\in\Delta^\Inter$ und jede Formel $F$ gilt:
\[ \Inter,\Zuweisung\{x\mapsto t\} \models F \qquad \text{gdw.}\qquad \Inter,\Zuweisung \models F\{x\mapsto t\}\]}

(ohne Beweis; einfach)
\bigskip

{\color{devilscss}\footnotesize \emph{Anmerkung:} Man kann ein entsprechendes Resultat auch für Nicht-Herbrand-Interpretationen zeigen. Dann muss man einfach den Term auf der linken Seite durch $t^{\Inter,\Zuweisung}$ ersetzen.}

\end{frame}

\begin{frame}\frametitle{Erfüllbar + Skolem = Erfüllbarkeit bei Herbrand}

\theobox{Satz: Ein Satz $F$ in Skolemform ist genau dann erfüllbar, wenn $F$ ein
Herbrandmodell hat.}\pause

\emph{Beweis:} $(\Leftarrow)$ ist klar, da Herbrandmodelle auch Modelle sind.\bigskip\pause

$(\Rightarrow)$ Sei $\Inter\models F$ ein Modell für $F$. Wir definieren eine Herbrandinterpretation $\Jnter$ indem wir festlegen:
\begin{itemize}
\item $p^\Jnter=\{\tuple{t_1,\ldots,t_n}\mid\tuple{t_1^\Inter,\ldots,t_n^\Inter}\in p^\Inter\}$\\
{\footnotesize Anm.: $t_i$ sind variablenfrei, daher ist $t_i^\Inter$ wohldefiniert}
\end{itemize}
Behauptung: \alert{$\Jnter$ ist ein Herbrandmodell von $F$}

\end{frame}

\begin{frame}\frametitle{Beweis (Fortsetzung)}

\emph{Behauptung:} \alert{$\Jnter$ ist ein Herbrandmodell von $F$}\bigskip

$F$ hat die Form $\forall x_1,\ldots, x_n.G$, wobei $G$ quantorenfrei ist.\pause
\begin{itemize}
\item Aus $\Inter\models F$ folgt also $\Inter,\Zuweisung\models G$ für jede Zuweisung $\Zuweisung$ für $\Inter$\pause
\item Speziell gilt also für alle $t_1,\ldots,t_n\in\Delta_F$: $\Inter,\{x_1\mapsto t_1^{\Inter},\ldots,x_n\mapsto t_n^{\Inter}\}\models G$\pause
\item Daraus folgt: $\Inter\models G\{x_1\mapsto t_1,\ldots,x_n\mapsto t_n\}$  (analog zu Lemma)\pause
\item Daraus folgt: $\Jnter\models G\{x_1\mapsto t_1,\ldots,x_n\mapsto t_n\}$\\
{\footnotesize(Für Atome $G$ direkt aus Definition; die Aussage kann leicht auf größere Boolsche Verknüpfungen von Atomen verallgemeinert werden -- formal durch strukturelle Induktion)}\pause
\item Es folgt: $\Jnter,\{x_1\mapsto t_1,\ldots,x_n\mapsto t_n\}\models G$ (Lemma)
\end{itemize}
Der Schluss gilt für alle $t_1,\ldots,t_n\in\Delta_F$, d.h. $\Jnter\models F$.\qed

\end{frame}


\sectionSlide{Prädikatenlogisches Schließen mit Aussagenlogik}

\begin{frame}\frametitle{Herbrand-Expansionen}

\defbox{Die \redalert{Herbrand-Expansion} $HE(F)$ einer Formel $F=\forall x_1,\ldots, x_n.G$ in Skolemform ist die Menge:
\[ HE(F)\defeq\big\{ G\{x_1\mapsto t_1,\ldots,x_n\mapsto t_n\}\mid t_1,\ldots,t_n\in\Delta_F\big\}\]
}

$HE(F)$ ist also die (möglicherweise unendliche) Menge von \ghost{variablen-}\\ freien Sätzen, die in Herbrandmodellen von $F$ gelten müssten.
\pause\bigskip

\redalert{Quantorenfreie Sätze = aussagenlogische Formeln:}
\begin{itemize}
\item $HE(F)$ enthält Formeln ohne Variablen, d.h. Boolsche Kombinationen geschlossener Atome
\item Geschlossene Atome können unabhängig voneinander wahr oder falsch sein, egal wie ihre genaue Struktur aussieht
\item Wir können sie also als "`ungewöhnlich benannte"' aussagenlogische Atome auffassen und die gesamte Formel aussagenlogisch interpretieren
\end{itemize}
\alert{$\leadsto$ $HE(F)$ als aussagenlogische Theorie}


\end{frame}

\begin{frame}\frametitle{Gödel, Herbrand, Skolem}

\theobox{Satz von Gödel, Herbrand \& Skolem: Eine Formel $F$ in Skolemform ist genau dann erfüllbar, wenn $HE(F)$ aussagenlogisch erfüllbar ist.}\pause

\emph{Beweis:} Wir zeigen, dass $F=\forall x_1,\ldots,x_n.G$ genau dann ein Herbrandmodell hat, wenn $HE(F)$ aussagenlogisch erfüllbar ist:\pause

\begin{itemize}
\item $\Inter$ ist ein Herbrandmodell von $F$\pause
\item gdw. $\Inter,\{x_1\mapsto t_1,\ldots,x_n\mapsto t_n\}\models G$ für alle $t_1,\ldots,t_n\in\Delta_F$\pause
\item gdw. $\Inter\models G\{x_1\mapsto t_1,\ldots,x_n\mapsto t_n\}$ für alle $t_1,\ldots,t_n\in\Delta_F$ (Lemma)\pause
\item gdw. für alle $H\in HE(F)$ gilt $\Inter\models H$\pause
\item gdw. $\Inter$ als aussagenlogisches Modell für $HE(F)$ angesehen werden kann.\qed
\end{itemize}

\end{frame}

\begin{frame}\frametitle{Satz von Herbrand}

Als Korollar der gezeigten Ergebnisse erhalten wir ein wichtiges Resultat:

\theobox{Satz: Eine Formel $F$ in Skolemform ist genau dann unerfüllbar, wenn
eine endliche Teilmenge von $HE(F)$ aussagenlogisch unerfüllbar ist.}

\pause\emph{Beweis:} Das Kontrapositiv des Satzes von Gödel, Herbrand \& Skolem besagt:\medskip

Eine Formel $F$ in Skolemform ist genau dann unerfüllbar, wenn $HE(F)$ aussagenlogisch unerfüllbar ist.\bigskip

Der Satz folgt nun, weil jede unerfüllbare aussagenlogische Formelmenge eine endliche
Teilmenge hat, die unerfüllbar ist: \alert{Kompaktheit der Aussagenlogik}, ohne Beweis\\[1ex]
{\tiny
\textcolor{devilscss}{Das Ergebnis kann aus der Vollständigkeit der Verallgemeinerung aussagenlogischer Resolution auf unendliche Modelle gefolgert werden (siehe Formale Systeme, WS 2017/2018, Vorlesung 23): wenn die leere Klausel endlich abgeleitet werden kann, dann nutzt man dazu nur endlich viele Klauseln der Eingabe; wenn die leere Klausel nicht endlich abgeleitet werden kann, dann erhält man aus der unendlichen Menge aller möglichen Ableitungen ein Modell, analog zum endlichen Fall.}

}\qed

\end{frame}

\begin{frame}\frametitle{Prädikatenlogik semi-entscheiden}

Das Ergebnis Herbrands ermöglicht bereits einen naiven Algorithmus zur Semi-Entscheidung von Unerfüllbarkeit in der Prädikatenlogik:\bigskip

\codebox{%
\emph{Gegeben:} Eine Formel $F$
\begin{itemize}
\item Wandle $F$ in Skolemform $F'$ um
\item Definiere eine Reihenfolge der Formeln in $HE(F')$: $F_1,F_2,F_3,\ldots$
\item Für alle $i\geq 1$:
	\begin{itemize}
	\item Prüfe ob die endliche Menge $\{F_1,\ldots, F_i\}$ aussagenlogisch unerfüllbar ist
	\item Falls ja, dann gib "`unerfüllbar"' aus; andernfalls fahre fort
	\end{itemize}
\end{itemize}}\medskip

Offenbar ist das \redalert{kein praktischer Algorithmus}, aber er zeigt Semi-Entscheidbarkeit

\end{frame}

\sectionSlide{Vollständigkeit der Resolution}

\begin{frame}\frametitle{Ansatz}

\alert{Herbrands Satz liefert uns auch eine Strategie zum Beweis der Vollständigkeit des Resolutionsalgorithmus}\bigskip

\emph{Wir wissen bereits:}
\begin{itemize}
\item Unerfüllbarkeit einer Klauselmenge zeigt sich in der Unerfüllbarkeit ihrer Herbrand-Expansion
\item Die Unerfüllbarkeit der Herbrand-Expansion kann man mit aussagenlogischer Resolution beweisen
\item Prädikatenlogische Resolution verallgemeinert aussagenlogische Resolution indem wir direkt mit Klauseln arbeiten, die noch Variablen enthalten
\end{itemize}\pause

\emph{Frage:} Kann man alle Schlüsse, die man auf expandierten Formeln aussagenlogisch erzeugen kann, auch direkt prädikatenlogisch (mit Variablen) erhalten?

\end{frame}

\begin{frame}\frametitle{Lifting-Lemma}

Wir zeigen: Ja, jeder aussagenlogische Schluss (auf der Expansion) kann auf einen prädikatenlogischen Schluss (auf den Klauseln mit Variablen) "`angehoben"' werden.\medskip

\theobox{Satz (Lifting-Lemma): Seien $K_1$ und $K_2$ prädikatenlogische Klauseln mit
Grundinstanzen $K'_1=K_1\sigma$ und $K'_2=K_2\sigma$.${^1}$\bigskip

Wenn $R'$ eine (aussagenlogische) Resolvente von $K'_1$ und $K'_2$ ist, dann gibt es
eine prädikatenlogische Resolvente $R$, welche $R'$ als Grundinstanz hat.
}

\color{devilscss}
{\footnotesize ${^1}$ Die Verwendung der selben Substitution für $K'_1$ und $K'_2$ ist keine Einschränkung, da wir durch Variantenbildung sicherstellen können, dass $K_1$ und $K_2$ keine Variablen gemein haben.

}

\end{frame}

\begin{frame}\frametitle{Lifting-Lemma: Beweis}

\theobox{Satz (Lifting-Lemma): Seien $K_1$ und $K_2$ prädikatenlogische Klauseln mit
Grundinstanzen $K'_1=K_1\sigma$ und $K'_2=K_2\sigma$.\smallskip

Wenn $R'$ eine (aussagenlogische) Resolvente von $K'_1$ und $K'_2$ ist, dann gibt es
eine prädikatenlogische Resolvente $R$, welche $R'$ als Grundinstanz hat.
}

\emph{Beweis:} Sei $A'\in K'_1$ das (geschlossene) Atom, über das resolviert wurde, d.h. $\neg A'\in K'_2$.\smallskip\pause

Sei $\mathcal{A}_1\defeq\{A\mid A\in K_1,\;A\sigma=A'\}$ und $\mathcal{A}_2\defeq\{A\mid \neg A\in K_2, A\sigma=A'\}$.\smallskip\pause

Dann ist $\sigma$ ein Unifikator für $\mathcal{A}_1\cup\mathcal{A}_2$.\\ Also hat $\mathcal{A}_1\cup\mathcal{A}_2$ einen allgemeinsten Unifikator $\theta$.\smallskip\pause

Sei $R$ die Resolvente von $K_1$ und $K_2$ bzgl. $\theta$.\smallskip\pause

Dann enthalten $R'$ und $R$ Instanzen der gleichen Literale, d.h. sie sind von der Form
$R'=\{L_1\sigma,\ldots,L_n\sigma\}$ und $R=\{L_1\theta,\ldots,L_n\theta\}$\smallskip\pause

Da $\theta$ allgemeinster Unifikator ist gibt es $\lambda$ mit $\sigma=\theta\circ\lambda$
und es gilt: $R\lambda=\{L_1\theta\lambda,\ldots,L_n\theta\lambda\}=\{L_1\sigma,\ldots,L_n\sigma\}=R'$\qed


\end{frame}


\begin{frame}[t]\frametitle{Vollständigkeit der Resolution (1)}

\theobox{Resolutionssatz: Sei $F$ eine prädikatenlogische Formel und $\mathcal{K}_i$ ($i\geq 0$) die vom Resolutionsalgorithmus ermittelten Klauselmengen. Dann sind die folgenden Aussagen äquivalent:
\begin{itemize}
\item $F$ ist unerfüllbar
\item Es gibt ein $\ell\geq 0$ mit $\bot\in\mathcal{K}_\ell$
\end{itemize}}\pause

\emph{Beweis (Vollständigkeit):} Sei $F$ unerfüllbar
\begin{itemize}
\item Dann ist $HE(F)$ unerfüllbar
\item Dann gibt es eine (endliche) aussagenlogische Resolutionsableitung von $\bot$ aus $HE(F)$
\item Die Ableitung erzeugt eine endliche Folge von Klauseln: $K_1',K_2',\ldots,K_{m-1}',K_m'=\bot$
\item \alert{Behauptung:} Jede Klausel $K_i'$ ist Grundinstanz einer Klausel $K_i$ die in $\mathcal{K}_\ell$ vorkommt für ein $\ell\geq 0$.
\item Für $i=m$ folgt daraus der Satz, denn $K_m'=\bot$ kann nur Grundinstanz von $\bot$ sein, d.h. $\bot\in\mathcal{K}_\ell$ für ein $\ell\geq 0$.
\end{itemize}

\end{frame}

\begin{frame}[t]\frametitle{Vollständigkeit der Resolution (2)}

% \theobox{Resolutionssatz: Sei $F$ eine prädikatenlogische Formel und $\mathcal{K}_i$ ($i\geq 0$) die vom Resolutionsalgorithmus ermittelten Klauselmengen. Dann sind die folgenden Aussagen äquivalent:
% \begin{itemize}
% \item $F$ ist unerfüllbar
% \item Es gibt ein $\ell\geq 0$ mit $\bot\in\mathcal{K}_\ell$
% \end{itemize}}

\emph{Beweis (Vollständigkeit):} \alert{Behauptung:} Jede Klausel $K_i'$ ist Grundinstanz einer Klausel $K_i$ die in $\mathcal{K}_\ell$ vorkommt für ein $\ell\geq 0$.
\medskip\pause

Aussage klar für $K'_i\in HE(F)$: in diesem Fall ist $K'_i$ ist Grundinstanz einer Klausel $K_i$ in der Klauselform von $F$ und in $\mathcal{K}_0$\medskip\pause

Restlicher Beweis durch Induktion über $i$:
\begin{itemize}
\item Induktionsannahme: Die Aussage gilt für alle $j<i$\pause
\item $K'_i$ ist Resolvente zweier Klauseln $K'_a$ und $K'_b$ mit $a,b<i$\pause
\item Laut Hypothese sind $K'_a$ und $K'_b$ also Instanzen von Klauseln $K_a$ und $K_b$ in einer Menge $\mathcal{K}_\ell$\pause
\item Laut Lifting-Lemma ist demnach $K'_i$ ebenfalls die Instanz einer Klausel $K_i$, die durch Resolution aus $K_a$ und $K_b$ entsteht\pause
\item Diese Resolvente $K_i$ ist also ebenfalls in einer Menge der Form $\mathcal{K}_{\ell'}$\qed
\end{itemize}

\end{frame}

\begin{frame}\frametitle{Kompaktheit}

Die Existenz von vollständigen und korrekten logischen Schließverfahren wie Resolution
ist eng verwandt mit einer grundsätzlichen Eigenschaft der Prädikatenlogik:\bigskip

\theobox{Satz (Endlichkeitssatz, Kompaktheitssatz): Falls eine unendliche Menge prädikatenlogischer Sätze $\mathcal{T}$ eine logische Konsequenz $F$ hat, so ist $F$ auch Konsequenz einer endlichen Teilmenge von $\mathcal{T}$.}\pause

\small
\emph{Beweis:} Die gegebene logische Konsequenz ist gleichbedeutend damit, dass
$\mathcal{T}\cup\{\neg F\}$ unerfüllbar ist.\medskip

Laut Resolutionssatz (Vollständigkeit) kann die Unerfüllbarkeit von $\mathcal{T}\cup\{\neg F\}$ nach endlich vielen Schritten durch Ableitung der leeren Klausel nachgewiesen werden.\medskip

Dabei können nur endlich viele Klauseln aus der Klauselform von $\mathcal{T}\cup\{\neg F\}$ verwendet worden sein. Laut Resolutionssatz (Korrektheit) folgt die Konsequenz also bereits aus einer endlichen Teilmenge von $\mathcal{T}$.\qed

\end{frame}
% 
% \begin{frame}\frametitle{Die Grenzen der Prädikatenlogik}
% 
% Kompaktheit zeigt uns auch erste Grenzen der Prädikatenlogik auf.
% \smallskip
% 
% \defbox{Eine logische Formel $F$ mit zwei freien Variablen $x$ und $y$ drückt den \redalert{transitiven Abschluss} einer binären Relation $r$ aus, wenn in jeder Interpretation $\Inter$ und für alle $\delta_1,\delta_2\in\Delta^\Inter$ gilt:\\[1ex]
% \narrowcentering{$ \Inter,\{x\mapsto \delta_1,y\mapsto\delta_2\}\models F \quad\text{gdw.}\quad \tuple{\delta_1,\delta_2}\in (r^\Inter)^* $}
% }\pause
% 
% \theobox{Satz: Es gibt keine prädikatenlogische Formel, die den transitiven Abschluss einer binären Relation ausdrückt.}\pause
% 
% \emph{Beweis:} Angenommen es gäbe so eine Formel $F$.\medskip
% 
% Dann ist die folgende unendliche Theorie unerfüllbar:
% \[\begin{array}{rll}
% \big\{ & F\{x\mapsto a,y\mapsto b\},\neg r(a,b), \neg \exists x_1.(r(a,x_1)\wedge r(x_1,b)),\\
% & \neg\exists x_1,x_2.(r(a,x_1)\wedge
% r(x_1,x_2)\wedge r(x_2,b)),\ldots & \big\}
% \end{array}\]
% Aber jede endliche Teilmenge der Theorie ist erfüllbar.
% Die Existenz der Theorie würde also dem Endlichkeitssatz widersprechen.\qed
% 
% \end{frame}

\begin{frame}\frametitle{Zusammenfassung und Ausblick}

Die prädikatenlogische Resolution ist ein vollständiges und korrektes Verfahren für die Unerfüllbarkeit logischer Formeln\bigskip

In gewissem Sinne ist Prädikatenlogik eine Kurzschreibweise für möglicherweise unendliche aussagenlogische Theorien
\bigskip

% Bei Beschränkung auf endliche Modelle gibt es kein vollständiges und korrektes Verfahren zum logischen Schließen -- dafür wird Erfüllbarkeit semi-entscheidbar\bigskip

% Auswertungsproblem auf endlichen Modellen = Anfragebeantwortung in Datenbanken

\anybox{yellow}{
Was erwartet uns als nächstes?
\begin{itemize}
\item Endliche Modelle und Datenbanken
\item Datalog
\item Gödel
\end{itemize}
}

\end{frame}


\begin{frame}[t]\frametitle{Bildrechte}

Folie \ref{frame_herbrand}, \ref{frame_herbrand_b}: Fotografie von Natasha Artin Brunswick, 1931, CC-By 3.0

\end{frame}


\end{document}
